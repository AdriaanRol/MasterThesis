\chapter{State Initialization}

\section{Gates used}
Gates are implemented according to \citet{Nielsen2010Quantum}.
\begin{equation}
    R_j(\theta)=e^{-i\theta \sigma_i}
\end{equation}

Where $\sigma_i$ are the pauli matrices:
\begin{equation}
\sigma_x=\frac{1}{2}\begin{bmatrix}0&1\\1&0\end{bmatrix}, \quad
\sigma_y=\frac{1}{2}\begin{bmatrix}0&-i\\i&0\end{bmatrix}\\,\quad
\sigma_z=\frac{1}{2}\begin{bmatrix}1&0\\0&1\end{bmatrix}
\end{equation}
Such that:
\begin{align*}
R_x(\pi/2)=&\frac{1}{\sqrt{2}}\begin{bmatrix}1&-i\\-i&1\end{bmatrix},
& R_x(-\pi/2)=&\frac{1}{\sqrt{2}}\begin{bmatrix}1&i\\i&1\end{bmatrix},
& R_x(\pi)=&\begin{bmatrix}0&-i\\-i&0\end{bmatrix}\\
%
R_y(\pi/2)=&\frac{1}{\sqrt{2}}\begin{bmatrix}1&-1\\1&1\end{bmatrix},
& R_y(-\pi/2)=&\frac{1}{\sqrt{2}}\begin{bmatrix}1&1\\-1&1\end{bmatrix},
& R_y(\pi)=&\begin{bmatrix}0&-1\\1&0\end{bmatrix}\\
%
R_z(\pi/2)=&\frac{1}{\sqrt{2}}\begin{bmatrix}(1-i)&0\\0&(1+i)\end{bmatrix},
& R_z(-\pi/2)=&\frac{1}{\sqrt{2}}\begin{bmatrix}(1+i)&0\\0&(1-i)\end{bmatrix},
& R_z(\pi)=&\begin{bmatrix}-i&0\\0&i\end{bmatrix}
\end{align*}

Furthermore, we assume that a conditional gate applied by resonantly decoupling the electron always rotates the carbon-13 spin along $\pm\hat{\mathrm{X}}$-axis.
We call the conditional $\pm\mathrm{x}$-gate $\bm{\mathrm{Ren}}$. Here $\rho_{i}$ is the density matrix of a qubit in state i, multiple subscripts denote multiple qubits where the initial qubit is always the electron.
\begin{align}
    \bm{\mathrm{Ren}} = \rho_{0}\otimes R_x(\pi/2)_n + \rho_{1}\otimes R_x(-\pi/2)_n
    \end{align}
    With for a two-qubit system:
    \begin{align}
    \bm{\mathrm{Ren}} = \frac{1}{\sqrt{2}}
        \begin{bmatrix}
            1& -i &0 &0 \\
            -i & 1 &0 &0 \\
            0 & 0 &1 &i \\
            0 & 0 &i &1
        \end{bmatrix}
\end{align}
An unconditional rotation is given by\cref{eq:tensor_identity}.
\begin{equation}
    R_j(\theta)_e = R_j(\theta) \otimes \bm{I}
    \label{eq:tensor_identity}
\end{equation}

\section{MBI-x initialization of a carbon-spin}

Where in order to initialize into the $\ket{X}_\mathrm{C}$-state we apply the circuit depicted in \cref{fig:gate_circuit_mbi_x-init}. In density matrix language this is can be represented up to the measurement in the following way
\begin{equation}
    \rho = R_x(\pi/2) \bm{\mathrm{Ren}} R_y(\pi/2)_e \rho_{0,\mathrm{mix}}
\end{equation}
% Furthermore, we assume that a conditional gate between the electron spin and a C13 spin always rotates the C13 spin along its $\pm\hat{x}-$axis, such that the so-called $\bm{\mathrm{Ren}}$ gate does the following to the system:
% \be
% \bm{\mathrm{Ren}} = \rho_{0,e}\otimes R_x(\pi/2)_n\rho_{1,e}\otimes R_x(-\pi/2)_n
% \ee
% where $\rho_{0,e}$ and $\rho_{1,e}$ are the projectors in $\ket{0}$ and $\ket{1}$ for the electron spin.



%Derivation of the operation the gate circuit does and how it initializes

% Same for the Tomography/Readout
