\chapter{Simulations and Calculations for Number of Addressable Carbons}
\label{chap:addressable_carbon_sims}

\section{Scaling of number of resonances with magnetic field}
Starting from \cref{eq:res_dip_loc}

\begin{equation}
    \tau = \frac{(2k+1)\pi}{2\omega_L+A_\parallel}
\end{equation}
In the limit where $\bm{\omega_L} \gg \bm{A}$ the number of resonances $K$ of a single carbon between $\tau = 0$ and $\tau = T_{\mathrm{DD},e}$ scales linear with the magnetic field: $N_k \propto \omega_L$.

At the same time the width of these resonances decreases quadratically with magnetic field (\cref{eq:res_dip_width}).
\begin{equation}
    \Delta = \frac{A_\perp }{2\omega_L^2}
\end{equation}
Combining these two we find that the number of resonances $N$ that fit between two orders to increase linearly with magnetic field.
\begin{eqnarray}
N_k -N_{k-1} = \frac{\tau_{k+1} -\tau_k} {\Delta} \\
N_k -N_{k-1} = \frac{2\pi}{2\omega_L +A_\parallel} \cdot \frac{2\omega_L ^2}{A\perp}\\
N_k -N_{k-1} = \frac{2\pi \omega_L}{A_\perp }
\end{eqnarray}



Meanwhile the time it takes to implement a $\pi/2$-gate is given by \cref{eq:number_of_pulses} where $N_{\pi/2} $ is the number of pulses required for a $\pi/2$-pulse.
\begin{equation}
    T_{\pi/2}= N_{\pi/2} \tau
    \label{eq:number_of_pulses}
\end{equation}
Using \cref{eq:contrast_single_carbon_spin}, and noting that $\hat{\mathrm{n_0}}$ and $\hat{\mathrm{n_1}}$ are anti-parallel at the resonance condition, we can find N to be:
\begin{eqnarray}
    \frac{\pi}{4} = \frac{N_{\pi/2} \phi}{2}\\
    N_{\pi/2}=\frac{\pi}{2\phi}
\end{eqnarray}
Where $\phi$ is given by \cref{eq:angle_term}.
\begin{equation}
    \phi =  \cos^{-1}\left(\cos(\tilde{\omega} \tau) \cos(\omega_L \tau)-\left(\frac{ A_ \parallel + \omega_L }{ \tilde{ \omega}}\right) \sin(\tilde{\omega} \tau)\sin(\omega_L \tau)\right)
    \label{eq:angle_term_appendix}
\end{equation}
In the limit where $\bm{\omega_L} \gg \bm{A}$, $\omega_L \approx \tilde{\omega}$ simplifying \cref{eq:angle_term_appendix} to:
\begin{eqnarray}
    \phi = \cos^{-1} \left(\cos^2(\omega_L \tau) -  \sin^2(\omega_L \tau) \right)\\
    \phi = \cos^{-1} \left(2 \cos(\omega_L \tau)\right)
\end{eqnarray}
% Reinserting this in \cref{eq:number_of_pulses} we find
% \begin{eqnarray}
%     T_{\pi/2} = \left(\frac{\pi}{2 \omega_L \tau}\right) \tau \\
%     T_{\pi/2} = \frac{\pi}{2 \omega_L}\\
% \end{eqnarray}
% Leading to the surprising conclusion that operations become faster when increasing magnetic field in the limit of
