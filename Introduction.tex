\chapter{Introduction}


%weakly coupled carbons in NV- environment provide a naturally occuring spin register making it a promising candidate for QC
% For large scale larger and  deterministically available spin registers are desirable
% here we show deterministic generation of entanglement paving the way for measurement based QEC.
%  Simultations indicate that it is possible to control even more Carbon atoms extending the register and performing full QEC

An essential feature for a scalable quantum computer is the ability to correct for quantum errors.
The Nitrogen-Vacancy center in diamond is a promising candidate for a scalable quantum computer.
In order to perform

This chapter will explain why building a quantum computer is cool/important/essential/someotherbigword
Explain why quantum error correction is essential in achieving this goal
Requirements to perform quantum error correction
Explain NV natural candidate
Requirements 3-5 qubits + entanglement.
This thesis will demonstrate how to address weakly coupled carbon spins and create entanglement between them.







% Paragraph: Quantum is cool because exponential scale up allowing previously unattainable problems


%The Nitrogen-Vacancy (NV) center in diamond is a promising candidate for a scalable quantum computer.
% An essential feature for building a scalable quantum computer is the ability to correct quantum errors.
% A basic form of Quantum Error Correction (QEC)  (cite references, either in NV or global)
% A logical next step is to perform QEC based on repeated parity measurements. (needs clarification).
% In order to perform measurement based QEC one needs to be able to initialize, control, read-out and entangle up-to 5 qubits.

The idea of using a quantum mechanical system to simulate physics was first explored by Feynman\cite{Feynman1982Simulating}. Because the Hilbert space(/state space?)  of a quantum mechanical system scales  exponentially with it's size one would need an exponentially large classical computer to simulate it's behavior. By manipulating a quantum mechanical system directly this scaling problem can be circumvented.

It was quantum simulation that eventually led to the idea of exploiting quantum effects to perform more efficient calculations but it wasn't until Shor's discovery of a remarkably efficient quantum algorithm for prime factorization in 1994\citep{Shor1994Algorithms}\comment{where should the number be placed?} that quantum information science really took off.

Shor's algorithm was the first example where a quantum computer can provide an exponential speedup over a classical computer. Shor's and other quantum algorithms allow solving classes of problems that were previously unsolvable, a well known example being the breaking of classical encryption codes.

%Look up TJ Ladd et al. 2010 nature. On exponential speedup and entanglement as key resource. Can I link entanglement as resource to the propagation of errors?

% Proof of concept exists (+ examples of basic algorithms) but key challenge is scaling it up
By now Shor's algorithm has been shown to work on a range of different small scale quantum computers \cite{Vandersypen2001Experimental} [Needs reference to Shor in different systems or basic algorithms in range of systems] but making a scalable quantum computer that can take full advantage of the exponential speedup proves elusive.

% Different systems and approaches, highlight node based and NV-

%  Key ingredients entanglement and dynamical operations (based on outcome of measurements)

%  Highlight Issue with errors as challenge and highlight threshold theorem

\section{Quantum Error Correction}
% Note Stabilizers not in there
% Explain idea of Quantum Error correction,

% Explain that Quantum Error Correction might allow for reaching threshold

% Key ingredients to performing Error Correction; Creating Entanglement and dynamical operations.

\section{Quantum Computing}
\begin{figure}[htbp]
    \centering
    \mbox{
    \Qcircuit @C=1em @R=.7em {
        &&&& \control &\cw & \cw &\cw & \cw &\cw & \cw &\cw & \control \cw &\\
        \lstick{\ket{0}_e}         & \qw & \targ &\targ & \meter \cwx &  && \lstick{\ket{0}_e}     &\qw & \targ & \targ & \meter &  \cw \cwx  &\lstick{\ket{0}_e} &\qw&\qw &\qw &\rstick{\ket{0}_e} \\
        \lstick{\ket{0}_{C1}}     &  \targ \qwx[1]& \ctrl{-1} &\qw & \qw &\qw&\qw& \qw &\qw & \qw &\qw & \qw & \multigate{2}{\mathrm{Corr.}} \cwx &\qw &\qw &  \targ \qwx[1] &\qw &\rstick{\ket{0}_{C1}}\\
        \lstick{\ket{\psi}_{C2}} &  \ctrl{1} & \qw &\ctrl{-2} & \qw &\qw&\qw&\qw  & \qw&\ctrl{-2} &\qw &\qw &\ghost{\mathrm{Corr.}}&\qw & \qw  &  \ctrl{1}&\qw &\rstick{\ket{\psi}_{C2}}\\
        \lstick{\ket{0}_{C3}}     & \targ   &\qw &\qw& \qw &\qw&\qw& \qw &\qw & \qw &\ctrl{-3} & \qw &\ghost{\mathrm{Corr.}}& \qw &\qw &  \targ&\qw &\rstick{\ket{0}_{C3}}
        \gategroup{1}{3}{5}{13}{.7em}{--}  \
        % \gategroup{1}{3}{5}{12}{.7em}{_\}}
        % \gategroup{1}{13}{5}{13}{.7em}{_\}}
        % \gategroup{1}{1}{5}{2}{.7em}{_\}}
        % \gategroup{1}{14}{5}{14}{.7em}{_\}}
        }
    }
    \caption{Gate circuit for three qubit error correction. First the state of $C2$ is encoded onto three qubits.
    To diagnose an error an ancilla qubit is used to perform two parity measurements on the encoded qubits.
    The result of these measurements is used to determine what gates to apply to correct for the error if it occurred.
    These error correcting operations (in the dahsed lines) can be repeated to continuously protect the encoded qubits against quantum errors.
    The three qubit error correction code can correct for one type of quantum error. }
    \label{fig:gate_circuit_3_qubit_QEC}
\end{figure}



\section{Weakly coupled carbons; a naturally occurring register }

Optical links \citep{Bernien2013Heralded,Pfaff2014Unconditional}
Nodes \citep{Taminiau2014Universal,Waldherr2014Quantum} (error correction)

Prominent approach to Scalable networks rely on repeated parity measurements (i.e. approach surface code. )

% Normally a source of errors (decoherence) weakly coupled carbons in the environment of the NV provide a naturally occuring register of qubits that can be used for QEC.

% Has been shown that they can be addressed and controlled (paper of Tim)

% Gate based QEC has been shown.
% Q: Why is this not full QEC? Don't need deterministic gates/FFWD?

% In this thesis we will show coherent control of weakly coupled carbons and deterministic entanglement creation.


% Notes, think of use of word parity and entanglement

%TODO_MAR: Revise part below
The Nitrogen Vacancy centre in diamond is a well investigated system\citep{Doherty2013NitrogenVacancy} and a promising candidate for quantum computation\citep{Childress2013Diamond}. In order to implement three qubit measurement based QEC we need three qubits plus ancillae that we can initialise, measure and conditionally perform operations on. These extra qubits are found in Carbon--13 atoms, which are normally a source of decoherence. These atoms can be addressed using a resonant decoupling sequence\citep{Taminiau2012Detection}.


\section{Goal of the project}
The Goal of the project is to create entanglement between weakly coupled carbon spins.
This is important because entanglement creation is a key component for QEC in particular and QI in general.

For this we need a b and c
