\chapter{Introduction}

\section{Quantum Computing}
% Paragraph: Quantum is cool because exponential scale up allowing previously unattainable problems

The idea of using a quantum mechanical system to simulate physics was first explored by Feynman\cite{Feynman1982Simulating}. Because the Hilbert space(/state space?)  of a quantum mechanical system scales  exponentially with it's size one would need an exponentially large classical computer to simulate it's behavior. By manipulating a quantum mechanical system directly this scaling problem can be circumvented.

It was quantum simulation that eventually led to the idea of exploiting quantum effects to perform more efficient calculations but it wasn't until Peter Shor's discovery of a remarkably efficient quantum algorithm for prime factorization in 1994 \cite{Shor1994Algorithms} that quantum information science really took off.

t wasn't until Peter Shor discovered a quantum algorithm for efficiently factoring large numbers in 1994\cite{Shor1994Algorithms} that it was realized how these devices could be used for other than academic purposes.
 that it was realized just how powerful these devices could be.
  that the full potential was realized
  that the field of quantum information science really took off.
It was Peter Shor's discovery of a remarkably efficient quantum algorithm for prime factorization in 1994 \cite{Shor1994Algorithms} that truly kick-started the field of quantum information.
It was Peter Shor's discovery of a remarkably efficient quantum algorithm for prime factorization in 1994 \cite{Shor1994Algorithms} that provides


dream for academic purposes
shor's algorithm kickstart


%Look up TJ Ladd et al. 2010 nature. On exponential speedup and entanglement as key resource.

% Proof of concept exists (+ examples of basic algorithms) but key challenge is scaling it up

% Different systems and approaches, highlight node based and NV-

%  Key ingredients entanglement and dynamical operations (based on outcome of measurements)

%  Highlight Issue with errors as challenge and highlight threshold theorem

\section{Quantum Error Correction}
% Note Stabilizers not in there
% Explain idea of Quantum Error correction,

% Explain that Quantum Error Correction might allow for reaching threshold

% Key ingredients to performing Error Correction; Creating Entanglement and dynamical operations.

\section{Weakly coupled carbons; a naturally occurring register }
% Normally a source of errors (decoherence) weakly coupled carbons in the environment of the NV provide a naturally occuring register of qubits that can be used for QEC.

% Has been shown that they can be addressed and controlled (paper of Tim)

% Gate based QEC has been shown.
% Q: Why is this not full QEC? Don't need deterministic gates/FFWD?

% In this thesis we will show coherent control of weakly coupled carbons and deterministic entanglement creation.


% Notes, think of use of word parity and entanglement
