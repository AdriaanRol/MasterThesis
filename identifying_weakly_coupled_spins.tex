\section{Identifying weakly-coupled carbon-pins}
To be able to control weakly coupled spins the interaction between these spins and the NV-center must be understood.
This section will discuss the effect of dynamical decoupling on the interaction between the NV-center and weakly coupled nuclear spins.
This knowledge is used to explain the features in the fingerprint of \cref{fig:FP} and identify several nuclear spins.

\subsection{The effect of dynamical decoupling on weakly coupled nuclear spins}

During dynamical decoupling the electronic state alternates between the $m_s = 0$ and $m_s =+1$ state, this causes the nuclear spin to alternately rotate around two distinct quantization axes (\cref{fig:quantax}).
When the electron is in the $m_s=0$ state each nuclear spin precesses about $\bm{\omega_L}$ with the Larmor frequency.
When the electron is in the $m_s=+1$ state there is a hyperfine interaction between the nucleus and the NV-center (\cref{eq:nuclear_hamiltonian_1}) and the spin precesses around $\bm{\tilde{\omega}}=\bm{\omega_L} +\bm{A}$, where $\bm{A} = A_\parallel \bm{\hat{z}} + A_\perp \bm{\hat{x}}$ \citep{Taminiau2012Detection}.


\begin{figure}[htbp]
\centering

        \begin{tikzpicture}
            \node[anchor=south west,inner sep=0] at (0,0) {\includegraphics[keepaspectratio,width=0.15\textwidth]{./img/QuantizationAxis.png}};
            \node[font=\huge, text = black] at (1.6,2.0)  {${\tilde{\omega}}$};
        \end{tikzpicture}
\caption{Flipping the electron spin from the  $m_s=0$ to the $m_s= +1$ state changes the quantization axis of nuclear spins. For  $m_s=0$ all nuclear spins precess about $\bm{\omega_L}$. For  $m_s=+1$ each spin precesses about a distinct axis $\bm{\tilde{\omega}}=\bm{\omega_L} +\bm{A}$.}
\label{fig:quantax}
\end{figure}

The result of a decoupling sequence is a net rotation around an axis $\bm{\hat{\mathrm{n_i}}}$ by an angle $\theta$.
Where $\bm{\hat{\mathrm{n_i}}}$ depends on the initial state of the electron and $\theta$ is proportional to the number of pulses $N$ \citep{Taminiau2012Detection}.
$\bm{\hat{\mathrm{n_i}}} =\bm{\hat{\mathrm{n_0}}}$ when the electron starts in $m_s = 0$ and $\bm{\hat{\mathrm{n_i}}} =\bm{\hat{\mathrm{n_1}}}$ when the electron starts in $m_s = +1$.

\begin{figure}[htbp]
    \begin{subfigure}[t]{0.49\textwidth}\centering
        \centering
        \caption{}
        \includegraphics{Img/unCond_rot_taminiau.pdf}
        \label{fig:uncond_rot}
    \end{subfigure}
    \begin{subfigure}[t]{0.49\textwidth}\centering
        \centering
        \caption{}
        \includegraphics{Img/Cond_rot_taminiau.pdf}
        \label{fig:cond_rot}
    \end{subfigure}
    \caption{\Cref{fig:uncond_rot} When the net rotation axes $\bm{\hat{\mathrm{n_0}}}$ and $\bm{\hat{\mathrm{n_1}}}$ point in the same direction the carbon experiences an unconditional rotation and cannot be controlled. \Cref{fig:cond_rot} When the net rotation axes $\bm{\hat{\mathrm{n_0}}}$ and $\bm{\hat{\mathrm{n_1}}}$ are anti-parallel the carbon experiences a conditional rotation, either around $+x$ or $-x$, and can be controlled. Figure from \citep{Taminiau2012Detection}.}
    \label{fig:conditional_and_unconditional_rotation}
\end{figure}

When the net rotation axes point in a different direction a conditional operation is executed during dynamical decoupling (\cref{fig:conditional_and_unconditional_rotation}).
In a dynamical decoupling spectroscopy contrast is lost when a conditional operation is executed.
To understand when this occurs it is useful to consider three different cases.
A weakly coupled carbon spin in: the \emph{trivial} regime where $A_\perp=0$, the \emph{basic} regime where $A_\perp \ll \omega_L$ and the \emph{complex} regime where $A_\perp \sim \omega_L$.
For a complete mathematical description of the response of nuclear spins to a dynamical decoupling spectroscopy the reader is referred to \cref{sec:mathematical_description_dd_spectro}.

\subsubsection{The trivial regime ($A_\perp=0$)}
Because there is no orthogonal component of the hyperfine the spin will precess around the $z$-axis independent of the initial electronic spin-state.
Therefore no conditional operation is possible and it is not possible to resolve such a spin in a dynamical decoupling spectroscopy.

\subsubsection{The basic regime ($A_\perp \ll \omega_L$)}
In the basic regime the net rotation axes are practically parallel and point in the $z$-direction for almost every $\tau$ except for a specific resonant condition for which the axes are anti-parallel.
This resonant condition is given by \cref{eq:res_dip_loc}, where $k$ is an integer, and has a Lorentzian shape in a dynamical decoupling spectroscopy.
The FWHM of the resonance is given by \cref{eq:res_dip_width}.

 \begin{equation}
\tau = \frac{(2k+1)\pi}{2 \omega_L + A_\parallel}
\label{eq:res_dip_loc}
\end{equation}
 \begin{equation}
\mathrm{FWHM_{DD}} = \frac{A_\perp}{2 \omega_L^2}
\label{eq:res_dip_width}
\end{equation}

\subsubsection{The complex regime ($A_\perp \sim \omega_L$)}

In the case where $\bm{\omega_L}$ and $\bm{A_\perp}$ are of comparable magnitude the net rotation axes $\bm{\hat{\mathrm{n_i}}}$ are strongly dependent on the initial electron-state for almost any $\tau$.
When a carbon is in the complex regime it is no longer possible to describe it as a narrow resonance in the dynamical decoupling spectroscopy.
The response is visible as a wide resonance with an oscillation on top of it.

\paragraph{}
Because the electron is not interacting with a single carbon but with a bath of carbon atoms the contrast $M$ is given by the product of all individual values $M_j$ for each individual spin $j$ (\cref{eq:prod_multiple_spins}).
When the responses of multiple carbons overlap contrast is quickly lost.
\begin{equation}
\label{eq:prod_multiple_spins}
    M = \prod_{j}{M_j}
\end{equation}

In order to distinguish an individual carbon-spin its response must not overlap with that of other carbon spins.
By sweeping the number of $\pi$-pules the response of an individual carbon can be distinguished from the response of multiple spins.
Only when an individual spin is being addressed is it possible to sweep the contrast of the dynamical decoupling spectroscopy to -1 by increasing the number of pulses.

\subsection{Identifying Individual Carbon-spins}
Going back to \cref{fig:FP} it is now possible to explain its features.

A broad feature with low coherence is clearly visible around $\tau/(4\tau_L) = m$ for odd $m$.
This feature is known as the spin-bath collapse and is caused by the response of multiple spins overlapping.

At the edges of the spin bath several sharp dips are visible. These most likely correspond to individual spins.
A first estimate of the hyperfine coupling to these spins can be made based on their location and width using \cref{eq:res_dip_loc,eq:res_dip_width}.

Between the spin-bath collapses there alternately appears an oscillation.
This oscillation is most likely caused by a spin in the complex regime.
It's position can be used to get a rough estimate for its hyperfine coupling.

By computing the responses for these estimated hyperfine parameters using \cref{eq:contrast_single_carbon_spin} a more accurate estimation can be made.
The parameters are varied until the computed response agrees with the data as well as possible.
Using this method 13 distinct carbon spins where identified.

The parameters of the 4 strongest coupled carbons are listed in \cref{tbl:HF_par} and their computed responses are visible as colored lines in \cref{fig:FP}.
All estimated hyperfine parameters and a link to the full fingerprint measurements can be found in \cref{chap:Fingerprint_data_appendix}.

\begin{table}[htbp]
\centering
    \begin{tabular}{cllll}
    Carbon & \quad \quad  $A_{\parallel} $ & \quad \quad $A_{\perp}$ \\ \hline
    1         & $2 \pi \cdot${ }30.0 kHz             & $2 \pi \cdot${ }80.0 kHz                \\
    2         & $2 \pi \cdot${ }27.0 kHz             & $2 \pi \cdot${ }28.5 kHz              \\
    3         & $2 \pi \cdot$-51.0 kHz          & $2 \pi \cdot$105.0 kHz              \\
    4         & $2 \pi \cdot${ }45.1 kHz           & $2 \pi \cdot${ }20.0 kHz                \\
    \end{tabular}
    \caption{Estimated hyperfine parameters for spins 1 to 4 in \cref{fig:FP}.}
    \label{tbl:HF_par}
\end{table}

