\section{Identifying weakly-coupled carbon-pins}
When discussing the Ramsey and the spin-echo experiment we have treated the NV-center as being affected by the spin-bath but not affecting it.
In reality the interaction works both ways and the NV-spin does affect the nuclear spins.
It is possible to probe these interactions using a dynamical decoupling spectroscopy.
A dynamical decoupling spectroscopy provides a type of fingerprint of the nuclear spin environment from which the the hyperfine interaction for the individual spins can be determined \citep{Taminiau2012Detection,Taminiau2014Universal}.

This section will discuss the effect of dynamical decoupling on the electron-nucleus dynamics.
This is used to explain the features in the fingerprint of \cref{fig:FP} and identify several individual nuclear spins.


\subsection{Dynamical decoupling spectroscopy}

In a dynamical decoupling spectroscopy experiment the electron is prepared in the $|X\rangle =\tfrac{1}{\sqrt{2}}\left( |0\rangle +|1\rangle \right) $ state.
It is subjected to a pulse sequence consisting of $N/2$ blocks of the form {$\tau - \pi -2\tau-\pi-\tau$}, where $\tau$ is a wait time and $\pi$ a $\pi$-pulse.
The experiment is concluded by measuring $\langle X\rangle $.
The fingerprint is the result of many repetitions for a range of inter-pulse delays $2\tau$.

Part of a dynamical decoupling spectroscopy result can be seen in \cref{fig:FP}.
When the electron spin performs an entangling operation on a spin in the environment coherence is lost when the electron spin is measured.
In a dynamical decoupling spectroscopy such an interaction is visible as a a decrease in the expectation value $\langle X\rangle$.

A broad collapse of the signal is clearly visible around $\tau/(4\tau_L) = m$ for odd $m$.
This broad collapse is caused by the electron performing an entangling gate on multiple spins.

At the edges of this collapse several sharp dips are visible.
These correspond to individual spins.

Between the broad collapses there alternately appears an oscillation.
This oscillation is caused by a spin in the complex regime.

The physical processes resulting in such a fingerprint will be discussed in the next section.

\begin{figure}[htbp]

    \begin{subfigure}[t]{\textwidth}\centering
        \caption{}
        \begin{tikzpicture}
            \node[anchor=south west,inner sep=0] at (0,0) {\includegraphics{Img/fingerprint16.pdf}};
            \node[font=\small, text = blue] at (4.05,1.7)  {1};
            \node[font=\small, text = green] at (9.3,2.9) {2};
            \node[font=\small, text = red] at (5.3,2.5) {3};
            \node[font=\small, text = cyan] at (4.0,3.35) {4};
            \node[font=\small, text = black] at (14.0,4.0) {$N=16$};
            % \draw[help lines,xstep=1,ystep=1] (0,0) grid (10,3);
            % \foreach \x in {1,2,...,10} { \node [anchor=north] at (\x,0) {\x}; }
            % \foreach \y in {1,2,...,3} { \node [anchor=east] at (0,\y) {\y}; }
        \end{tikzpicture}
        \label{fig:FP16}
    \end{subfigure}

    \begin{subfigure}[t]{\textwidth}\centering

    \caption{}
    \begin{tikzpicture}
        \node[anchor=south west,inner sep=0] at (0,0) {\includegraphics{Img/fingerprint32.pdf}};
        \node[font=\small, text = blue] at (11.2,1.9) {1};
        \node[font=\small, text = blue] at (4.0,2.8)  {1};
        \node[font=\small, text = green] at (6.6,1.8) {2};
        \node[font=\small, text = red] at (7.4,1.8) {3};
        \node[font=\small, text = cyan] at (4.05,2.5) {4};
        \node[font=\small, text = black] at (14.0,4.0) {$N=32$};
        % \draw[help lines,xstep=1,ystep=1] (0,0) grid (10,3);
        % \foreach \x in {1,2,...,10} { \node [anchor=north] at (\x,0) {\x}; }
        % \foreach \y in {1,2,...,3} { \node [anchor=east] at (0,\y) {\y}; }
    \end{tikzpicture}
    \label{fig:FP32}
    \end{subfigure}
    \caption{Part of a dynamical decoupling spectroscopy experiment performed at $B_z = 304\,\mathrm{G}$, $\tau_L =3.07 \, \mu \mathrm{s} $.
    Black lines correspond to data. Colored lines represent computed responses of carbon spins.
    \subref{fig:FP16} $N = 16$ pulses; \subref{fig:FP32} $N=32$ pulses.
    Contrast is lowered when the decoupling sequence performs an entangling operation on a spin in the environment.
    A reference to the full dynamical decoupling dataset can be found in \cref{chap:Fingerprint_data_appendix}.
    Responses were calculated using \cref{eq:contrast_single_carbon_spin} with hyperfine parameters from \cref{tbl:HF_par}. }
    \label{fig:FP}
\end{figure}





\subsection{The effect of dynamical decoupling on nuclear spins}

During dynamical decoupling the electronic state alternates between the $m_s = 0$ and $m_s =+1$ state, this causes the nuclear spin to alternately rotate around two distinct quantization axes (\cref{fig:quantax}).
When the electron is in the $m_s=0$ state each nuclear spin precesses about $\bm{\omega_L}$ with the Larmor frequency given by \cref{eq:nuclear_larmor} .
When the electron is in the $m_s=+1$ state there is a hyperfine interaction between the nucleus and the NV-center (\cref{eq:nuclear_hamiltonian_1}) and the spin precesses around $\bm{\tilde{\omega}}=\bm{\omega_L} +\bm{A}$, where $\bm{A} = A_\parallel \bm{\hat{z}} + A_\perp \bm{\hat{x}}$ \citep{Taminiau2012Detection}.


\begin{figure}[htbp]
\centering

        \begin{tikzpicture}
            \node[anchor=south west,inner sep=0] at (0,0) {\includegraphics[keepaspectratio,width=0.15\textwidth]{./img/QuantizationAxis.png}};
            \node[font=\huge, text = black] at (1.6,2.0)  {${\tilde{\omega}}$};
        \end{tikzpicture}
\caption{Flipping the electron spin from the  $m_s=0$ to the $m_s= +1$ state changes the quantization axis of nuclear spins. For  $m_s=0$ all nuclear spins precess about $\bm{\omega_L}$. For  $m_s=+1$ each spin precesses about a distinct axis $\bm{\tilde{\omega}}=\bm{\omega_L} +\bm{A}$ due to the hyperfine interaction.}
\label{fig:quantax}
\end{figure}

The result of a decoupling sequence is a net rotation around an axis $\bm{\hat{\mathrm{n}}_i}$ by an angle $\theta$.
Where $\bm{\hat{\mathrm{n}}_i}$ depends on the initial state of the electron and $\theta$ is proportional to the number of pulses $N$ \citep{Taminiau2012Detection}.
$\bm{\hat{\mathrm{n}}_i} =\bm{\hat{\mathrm{n}}_0}$ when the electron starts in $m_s = 0$ and $\bm{\hat{\mathrm{n_i}}} =\bm{\hat{\mathrm{n}}_1}$ when the electron starts in $m_s = +1$.

\begin{figure}[htbp]
    \begin{subfigure}[t]{0.49\textwidth}\centering
        \centering
        \caption{}
        \includegraphics{Img/unCond_rot_taminiau.pdf}
        \label{fig:uncond_rot}
    \end{subfigure}
    \begin{subfigure}[t]{0.49\textwidth}\centering
        \centering
        \caption{}
        \includegraphics{Img/Cond_rot_taminiau.pdf}
        \label{fig:cond_rot}
    \end{subfigure}
    \caption{\subref{fig:uncond_rot} When the net rotation axes $\bm{\hat{\mathrm{n}}_0}$ and $\bm{\hat{\mathrm{n}}_1}$ point in the same direction the carbon experiences an unconditional rotation. \subref{fig:cond_rot} When the net rotation axes $\bm{\hat{\mathrm{n}}_0}$ and $\bm{\hat{\mathrm{n}}_1}$ are anti-parallel the carbon experiences a conditional rotation, either around $+x$ or $-x$. Figure from \citet{Taminiau2012Detection}.}
    \label{fig:conditional_and_unconditional_rotation}
\end{figure}

When the net rotation axes point in a different direction a conditional operation is executed during dynamical decoupling (\cref{fig:conditional_and_unconditional_rotation}).
In a dynamical decoupling spectroscopy contrast is lowered when a conditional operation is executed.
To understand when this occurs it is useful to consider two different regimes for carbon spins: the basic regime where $A_\perp \ll \omega_L$ and the complex regime where $A_\perp \sim \omega_L$.
For a mathematical description of the result of a dynamical decoupling spectroscopy the reader is referred to \cref{sec:mathematical_description_dd_spectro}.

\subsubsection{The basic regime ($A_\perp \ll \omega_L$)}
In the basic regime the net rotation axes are practically parallel and point in the $z$-direction for almost every $\tau$ except for a specific resonant condition for which the axes are anti-parallel.
This resonant condition is given by \cref{eq:res_dip_loc}:
 \begin{equation}
\tau = \frac{(2k+1)\pi}{2 \omega_L + A_\parallel}
\label{eq:res_dip_loc}
\end{equation}
Where $k$ is an integer, and has a Lorentzian shape in a dynamical decoupling spectroscopy.
These sharp resonances are the origin of the sharp dips in the dynamical decoupling spectroscopy.
The FWHM of the resonance is given by \cref{eq:res_dip_width}:
 \begin{equation}
\mathrm{FWHM_{DD}} = \frac{A_\perp}{2 \omega_L^2}
\label{eq:res_dip_width}
\end{equation}


\subsubsection{The complex regime ($A_\perp \sim \omega_L$)}

In the case where $\bm{\omega_L}$ and $\bm{A_\perp}$ are of comparable magnitude the net rotation axes are strongly dependent on the initial electron-state for almost any $\tau$.
When a carbon is in this regime it is no longer possible to describe it as a narrow resonance in the dynamical decoupling spectroscopy.
The response is visible as a wide resonance with an oscillation on top of it, spin 3 in \cref{fig:FP} is an example of such a spin.

\paragraph{ }
When the response of multiple carbon spins in a dynamical decoupling spectroscopy overlaps, the electron performs an entangling operation on all these carbons.
Because of this the measured contrast in a dynamical decoupling spectroscopy goes down rapidly when the responses of multiple carbons overlap.
This makes it hard to resolve individual spins when a spin has a broad response, such as a spin in the complex regime, or if there are several spins with similar hyperfine couplings.


\subsection{Identifying Individual Carbon-spins}
Using the previous section it is possible to identify individual spins.
A first estimate of the hyperfine coupling to these spins can be made based on the location and width of dips in the dynamical decoupling spectroscopy using \cref{eq:res_dip_loc,eq:res_dip_width}.
By computing the responses for the hyperfine parameters using \cref{eq:contrast_single_carbon_spin} the estimation can be improved.
Using this method 13 distinct carbon spins were identified.

The parameters of the 4 strongest coupled carbons are listed in \cref{tbl:HF_par} and their computed responses are visible as colored lines in \cref{fig:FP}.
All estimated hyperfine parameters and a link to the full dynamical decoupling dataset can be found in \cref{chap:Fingerprint_data_appendix}.

\begin{table}[htbp]
\centering
    \caption{Estimated hyperfine parameters for spins 1 to 4 in \cref{fig:FP}.}
    \begin{tabular}{cllll}
    Carbon & \quad \quad  $A_{\parallel} $ & \quad \quad $A_{\perp}$ \\ \hline
    1         & $2 \pi \cdot${ }30.0 kHz             & $2 \pi \cdot${ }80.0 kHz                \\
    2         & $2 \pi \cdot${ }27.0 kHz             & $2 \pi \cdot${ }28.5 kHz              \\
    3         & $2 \pi \cdot$-51.0 kHz          & $2 \pi \cdot$105.0 kHz              \\
    4         & $2 \pi \cdot${ }45.1 kHz           & $2 \pi \cdot${ }20.0 kHz                \\
    \end{tabular}
    \label{tbl:HF_par}
\end{table}

