\chapter{Controlling Weakly-coupled Carbon Spins}
[Note intro needs to be rewritten once chapter is done]\\
Now that we have identified several promising carbon spins we will go into controlling them.

In order to demonstrate entanglement we must be able to initialize the system, perform a parity measurement to create the entanglement and make a tomography to demonstrate it.
This chapter will explain what operations need to be performed.

This chapter attempts to explain first how we can control a carbon by explaining how the two different rotation axes from the last chapter can be used to implement a conditional +-X gate. A gate similar to a CNOT gate.
Now that we understand how to implement a cnot gate we will use this to perform an experiment that we call a carbon ramsey on an unitialized state.
AFterwards we will use this to show that we can initialize and readout by showing an initialized carbon ramsey. (1buit tomo (ramsey with more T +Z-RO?).
From this we will also determine $T_{2 \mathrm{C}}^*$.

The next section will consider two qubit control.
It will start by explaining how we can perform a parity measurement. This measurement allows us to perform a two qubit tomography and create entanglement by projection.
We will then show



\section{Initialization and readout}
The $\pm\mathrm{x}$-gate can be used to initialize a carbon-spin.
To initialize a carbon-spin  in the $ \ket{X}$-state the gate-circuit as depicted in \cref{fig:gate_circuit_mbi_x-init} is implemented.

After the electronic-spin is brought in the $\ket{X}$-state the two-qubit system can be described by the tensor product of two density matrices:
\begin{equation}
    \rho_X \otimes \rho_m = \rho_X \otimes \rho_{X} +\rho_X \otimes \rho_{-X}
\end{equation}
By applying the $\pm{\mathrm{x}}$-gate  the electronic-spin picks up a phase depending on the nuclear spin-state:
\begin{equation}
     \rho_Y \otimes \rho_{X} +\rho_{-Y} \otimes \rho_{-X}
    \label{eq:density_after_Ren}
\end{equation}

The effect of the $\pm{\mathrm{x}}$-gate can be made more easily understood by treating the carbon-spin as being either in the $\ket{X}$ or $\ket{-X}$-state. Treating both as separate cases below.
\begin{align}
    \quad &\ket{X}\ket{X} &\vee \quad &  \quad\ket{X}\ket{-X}\\
    \quad &\frac{\ket{0}+\ket{1}}{\sqrt{2}}\ket{X} &\vee \quad & \quad \frac{\ket{0}+\ket{1}}{\sqrt{2}}\ket{-X} \\
    \quad &\frac{e^{-i\pi/4}\ket{0}+e^{i\pi/4}\ket{1}}{\sqrt{2}}\ket{X} &\vee \quad & \quad \frac{e^{i\pi/4}\ket{0}+e^{-i\pi/4}\ket{1}}{\sqrt{2}}\ket{-X}\\
    \quad &\ket{Y}\ket{X} &\vee \quad & \quad \ket{-Y}\ket{-X}
\end{align}
The final x-pulse is used to read out the electron. By conditionalizing the experiment on getting the outcome $\ket{0}_e$ the carbon is initialized in the $\ket{X}_C$-state.
\begin{equation}
    \rho_0 \otimes \rho_{\mathrm{X}} + \rho_1 \otimes \rho_{\mathrm{-X}}
\end{equation}

\begin{figure}[htbp]
    \centering
    \begin{subfigure}[t]{0.49\textwidth}
    \centering
    \caption{}
    \mbox{
        \Qcircuit @C=1em @R=.7em {
        \lstick{\ket{0}_e}                        & \gate{\mathrm{y}}  & \ctrl{1}       & \gate{\mathrm{x}} &\qw          &  \meter \\
        \lstick{\rho_\mathrm{m}}         & \qw              &  \gate{\pm \mathrm{x}}     & \qw    & \qw   & \qw}}
    \label{fig:gate_circuit_mbi_x-init}
    \end{subfigure}
    \begin{subfigure}[t]{0.49\textwidth}
        \centering
        \caption{}
        \mbox{
        \Qcircuit @C=1em @R=.7em {
            \lstick{\ket{0}_e} & \gate{\mathrm{y}}  & \ctrl{1} & \gate{\mathrm{x}} &\ctrl{1} &  \meter \\
            \lstick{\rho_\mathrm{m}}& \qw&  \gate{\pm \mathrm{x}}     & \qw    & \gate{\mp \mathrm{y}}    & \qw}}
        \label{fig:gate_circuit_mbi_swap-init}
    \end{subfigure}
    \caption{\Cref{fig:gate_circuit_mbi_x-init} MBI-based initialization into $\pm \ket{\mathrm{X}}$. Initializes the carbon into $\ket{X}_\mathrm{C} $ when $\ket{0}_e$ is measured and into $\ket{-X}_\mathrm{C} $ when $\ket{1}_e$ is measured for the electron.
    \Cref{fig:gate_circuit_mbi_swap-init} MBI-swap initialization into $ \ket{\mathrm{0}}$. Initializes the carbon into $\ket{0}_\mathrm{C} $ regardless of the electronic spin-state measured.}
    \label{fig:gate_circuit_initialization}
\end{figure}

The sequence can be extended to a swap type initialization by implementing an additional $\mp{\mathrm{y}}$-gate.
The effect of the extra y-gate is that both the  $\ket{X}_\mathrm{C} $ and the  $\ket{-X}_\mathrm{C} $-state are rotated to the  $\ket{0}_\mathrm{C} $-state, effectively swapping the mixed state from the carbon to the electron.
By applying a readout the electron can be reinitialized.

\begin{figure}[htbp]
    \centering
    \includegraphics{Img/RO_and_init_C1.pdf}
    \caption{Demonstration of carbon control. Figure shows carbon-1 }
    \label{fig:control_demo}
\end{figure}


% Want to explain here how frequency is determined down to Hz accuracy.
\section{Parity measurement}

Initialization by conditionalizing on parity measurement and Readout parities.
Measurement projects into parity or no parity.
\begin{figure}[htbp]
    \centering
\mbox{
\Qcircuit @C=1em @R=.7em {
\lstick{\ket{0}_e} & \gate{\mathrm{y}}  & \ctrl{1} &  \ctrl{2} & \gate{\mathrm{y}}  & \ctrl{1} &  \ctrl{2} &  \meter &\qw\\
\lstick{\ket{0}_{C1}} & \qw&  \gate{\pm \mathrm{x}}  &\qw  &\qw  & \gate{\pm \mathrm{x}} & \qw   &\qw&\qw& \\
\lstick{\ket{0}_{C2}}& \qw& \qw  & \gate{\pm \mathrm{x}}    & \qw    &\qw& \gate{\pm \mathrm{x}}    & \qw &\qw&}}
    \caption{General parity RO}
    \label{fig:gate_circuit_general_Parity_RO}
\end{figure}

\section{Demonstrating entanglement}

\subsection{Tomography}
\subsection{Result}
\begin{figure}[htbp]
    \begin{subfigure}[t]{0.49\textwidth}\centering
        \caption{}
        \includegraphics{Img/uu-no-parity.pdf}
        \label{fig:uu-init }
    \end{subfigure}
    \begin{subfigure}[t]{0.49\textwidth}\centering
        \caption{}
        \includegraphics{Img/uu-XX-parity.pdf}
        \label{fig:uu-XX}
    \end{subfigure}

    \begin{subfigure}[t]{0.49\textwidth}\centering
        \caption{}
        \includegraphics{Img/ud-no-parity.pdf}
        \label{fig:ud-init }
    \end{subfigure}
    \begin{subfigure}[t]{0.49\textwidth}\centering
        \caption{}
        \includegraphics{Img/ud-XX-parity.pdf}
        \label{fig:ud-XX}
    \end{subfigure}
    \caption{ Preliminary data
    \cref{fig:uu-init } Fidelity to guess  = 0.814256440281 +/- 0.0168202298443\\
    \cref{fig:uu-XX}  Fidelity to guess  = 0.765954332553 +/- 0.0174228538115\\
    \cref{fig:ud-init } Fidelity to guess  = 0.809865339578 +/- 0.0168850158243\\
    \cref{fig:ud-XX}  Fidelity to guess  = 0.760831381733 +/- 0.0174356869379
    }
    \label{fig:2qubitTomos}
\end{figure}

% What bars to include?
% Experiment that initializes both carbons + bar graph? -> to explain how to read + to explain

% 20140810 122740 Z Z init
%  20140810 132412  YYparity
%
% \begin{figure}[htbp]
%     \centering
% \mbox{
% \Qcircuit @C=1em @R=.7em {
% \lstick{\ket{0}_e} & \gate{\mathrm{y}}  & \ctrl{1} &  \ctrl{2} & \gate{y}  &  \meter &\qw\\
% \lstick{\ket{0}_{C1}} & \qw&  \gate{\pm \mathrm{x}}  &\qw  & \qw       &\qw&\qw& \\
% \lstick{\ket{0}_{C2}}& \qw& \qw  & \gate{\pm \mathrm{x}}    & \qw      &\qw&\qw&}}
%     \caption{XX-parity-measurement}
%     \label{fig:gate_circuit_XX-parity-measurement}
% \end{figure}



%  Should emphasize difficulty in seperating initialization and RO fidelity, what is not working? Is it working?


