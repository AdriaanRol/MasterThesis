\chapter{Controlling Weakly-coupled Carbon Spins}
The creation of entanglement is an essential capability for QEC in particular and quantum-computation in general.
This chapter demonstrates how weakly coupled nuclear spins can be initialized and read-out and how this can be used to generate entanglement between them.
The creation of entanglement is demonstrated with a quantum state tomography.

\section{Initialization and readout of single spins}
\label{sec:carbon_init_and_readout}
A weakly coupled spin can be initialized by conditionalizing on a readout result, similar to how nitrogen-MBI works.
A weakly coupled carbon spin can be read-out by using the $\pm \mathrm{x}$-gate to entangle the phase of the electronic-spin with the state of the nuclear spin and reading out the phase of the electronic spin.

%Explain Readout.
The gates used to initialize a weakly coupled nuclear spin are depicted in \cref{fig:gate_circuit_initialization}.
The circuit depicted in \cref{fig:gate_circuit_mbi_x-init} is used to perform a read-out along the $x$-axis.
By applying a $\pm y$-gate on the carbon before the initial $y$-pulse the state can be read out along the $z$-axis.

%Explain Initialization.
\begin{figure}[htbp]
    \centering
    \begin{subfigure}[t]{0.49\textwidth}
    \centering
    \caption{}
    \mbox{
        \Qcircuit @C=1em @R=.7em {
        \lstick{\ket{0}_e}                        & \gate{\mathrm{y}}  & \ctrl{1}       & \gate{\mathrm{x}} &\qw          &  \meter \\
        \lstick{\rho_\mathrm{m}}         & \qw              &  \gate{\pm \mathrm{x}}     & \qw    & \qw   & \qw}}
    \label{fig:gate_circuit_mbi_x-init}
    \end{subfigure}
    \begin{subfigure}[t]{0.49\textwidth}
        \centering
        \caption{}
        \mbox{
        \Qcircuit @C=1em @R=.7em {
            \lstick{\ket{0}_e} & \gate{\mathrm{y}}  & \ctrl{1} & \gate{\mathrm{x}} &\ctrl{1} &  \meter \\
            \lstick{\rho_\mathrm{m}}& \qw&  \gate{\pm \mathrm{x}}     & \qw    & \gate{\mp \mathrm{y}}    & \qw}}
        \label{fig:gate_circuit_mbi_swap-init}
    \end{subfigure}
    \caption{\Cref{fig:gate_circuit_mbi_x-init} MBI-based initialization into $\pm \ket{\mathrm{X}}$. Initializes the carbon into $\ket{X}_\mathrm{C} $ when $\ket{0}_e$ is measured and into $\ket{-X}_\mathrm{C} $ when $\ket{1}_e$ is measured for the electron.
    \Cref{fig:gate_circuit_mbi_swap-init} MBI-swap initialization into $ \ket{\mathrm{0}}$. Initializes the carbon into $\ket{0}_\mathrm{C} $ regardless of the electronic spin-state measured.}
    \label{fig:gate_circuit_initialization}
\end{figure}


% To initialize the nuclear spin
The initial pulse in \cref{fig:gate_circuit_mbi_x-init} brings the electronic spin in $\ket{X}_e$.
When the nuclear spin starts in the mixed state the system can be described by the tensor product of two density matrices:
\begin{equation}
    \rho_X \otimes \rho_m = \rho_X \otimes \rho_{X} +\rho_X \otimes \rho_{-X}
\end{equation}
By applying the $\pm{\mathrm{x}}$-gate  the electronic-spin picks up a phase depending on the nuclear spin-state:
\begin{equation}
     \rho_Y \otimes \rho_{X} +\rho_{-Y} \otimes \rho_{-X}
\end{equation}
By reading out the electronic spin along the $y$-axis the nuclear spin is projected into the $\ket{X}_C$ or $\ket{-X}_C$-state.
By conditionalizing on a positive readout result the state can be initialized into the $\ket{X}_C$-state.

By adding a second conditional gate (\cref{fig:gate_circuit_mbi_swap-init}) the nuclear spin is brought into the $\ket{0}_C$-state regardless of the measurement outcome for the electronic spin.

\paragraph{}
\Cref{fig:single_qubit_initialization} demonstrates initialization and readout.
In \cref{fig:carbon_init_x} carbon-1 is initialized into the $\ket{X}_C$-state and in \cref{fig:carbon_init_Z} it is initialized into the $\ket{0}_C$-state.
This is done by implementing the circuits depicted in \cref{fig:gate_circuit_initialization} and conditionalizing on a positive electron readout.

The carbon is read out using the same circuits.
The blue points correspond to $x$-readout, the green points to $y$-readout and the red-points to the $z$-readout.
The phase is sweeped to demonstrate that the readouts function as intended.
For a qubit in the $\ket{0}$-state the initial phase is undefined.

The combined fidelity of readout and initialization to the desired state is: $F(\ket{X}) = 90.57 \pm 0.85 \% $ for the $\ket{X}$-state and $F(\ket{Z}) = 93.00 \pm 0.31 \%$ for the $\ket{Z}$-state.
% Add description of features if still have time after finished today :)

\begin{figure}[htbp]
    \begin{subfigure}[t]{0.49\textwidth}\centering
        \caption{}
        \includegraphics{Img/RO_and_init_C1_X.pdf}
        \label{fig:carbon_init_x}
    \end{subfigure}
        \begin{subfigure}[t]{0.49\textwidth}\centering
        \caption{}
        \includegraphics{Img/RO_and_init_C1_Z.pdf}
        \label{fig:carbon_init_Z}
    \end{subfigure}
    \caption{Demonstration of carbon initialization and readout. In \cref{fig:carbon_init_x} carbon-1 is initialized in $\ket{X}$ and read-out. In \cref{fig:carbon_init_Z} carbon-1 is initialized in $\ket{0}$. Colored points correspond to readouts in different bases, blue to $x$-readout, green to $y$-readout and red to $z$-readout.}
    \label{fig:single_qubit_initialization}
\end{figure}


\section{The parity measurement}
Entanglement between two qubits can be created by performing a parity measurement on them and conditionalizing on the outcome.

A parity measurement does not measure the state of two qubits but measures if  two qubits are the same in a certain basis.
An example is the XX-parity measurement.
The XX-parity measurement returns a positive result if the the two qubits are the same in the $x$-basis and a negative result if the two are opposite.
That is it returns a positive result if the state is $\ket{X,X}$ or $\ket{-X,-X}$ and a negative result if the state is $\ket{X,-X}$ or $\ket{-X,X}$.
In general a two qubit parity operator has 2 eigenvalues, both are twofold degenerate.

The XX-parity measurement can be implemented on a weakly coupled carbon spins using the circuit depicted in \cref{fig:gate_circuit_general_Parity_RO}.
A parity measurement is very similar to the regular readout depicted in \cref{fig:gate_circuit_mbi_x-init}.
Once the electron is brought into a superposition the electron picks up phase when the $\mp \mathrm{x}$-gate is applied.
$+\pi/2$-phase when the carbon is in $\ket{+X}$ and $-\pi/2$-phase when the carbon is in $\ket{-X}$.
This is done for both carbons, when both carbons are in the same $x$-state the electron will pick up $\pi$-phase.
When they do not give the same result the phase cancels.
By reading out the electronic-spin along $x$ the parity is measured.
It should be noted that because we use a $\pm \mathrm{x}$-gate instead of a CNOT-gate an additional $\pi/2$-phase is added to the carbon states compared to a regular parity measurement.

\begin{figure}[htbp]
    \centering
\mbox{
\Qcircuit @C=1em @R=.7em {
\lstick{\ket{0}_e} &  \gate{\mathrm{y}}  & \ctrl{1} &  \ctrl{2} & \gate{\mathrm{y}}  &  \meter &\qw\\
\lstick{\ket{\psi}_{C1}} &  \qw & \gate{\pm \mathrm{x}}  &\qw   &  \qw   &\qw&\qw \\
\lstick{\ket{\psi}_{C2}}   & \qw   & \qw    & \gate{\pm \mathrm{x}}   &\qw & \qw &\qw}}
    \caption{Gate circuit for a XX-parity measurement. }
    \label{fig:gate_circuit_general_Parity_RO}
\end{figure}

\section{Quantum state tomography}
To demonstrate entanglement, entanglement must not only be created but it must also be verified that the entangled state is created.
This can be done by performing a quantum state tomography.
In a quantum state tomography the density matrix of a quantum state is reconstructed by repeatedly preparing the same state and gathering measurement statistics in different bases.

An arbitrary matrix can be described as a weighted sum of the Pauli-matrices and the Identity as in: \cref{eq:pauli}.
\begin{equation}
    \rho = I + \sum_{i,j} a_{i,j} \sigma_i \otimes \sigma_j
    \label{eq:pauli}
\end{equation}
By measuring the coefficients of \cref{eq:pauli}  the density matrix $\rho$ can be reconstructed completely.


\subsection{Readout}
To measure the coefficients of \cref{eq:pauli} single and multi qubit measurements are needed.
Single qubit measurements were described in \cref{sec:carbon_init_and_readout}.
The two qubit measurements required are very similar to parity measurements but do not need to preserve the state after the measurement.

The parity measurement depicted in \cref{fig:gate_circuit_general_Parity_RO} can be used to measure the XX-parity.
By changing the phase of one or two of the two $\pm \mathrm{x}$ gates Y-parities can be measured.
By applying a $\mp \mathrm{y}$ to one of the two carbons before the initial Y-pulse a Z-parity can be measured, care must be taken however that the phase difference between this gate and the $\pm \mathrm{x}$ on the corresponding carbon is $90^\circ$.

% How readout is implemented
\subsection{Initialization and tomography of multiple weakly coupled spins}
Before entanglement can be created the system must be initialized.
We verify that we have correctly initialized the qubits by performing a tomography.
The results are compared to the expected coefficients for the ideal case.

\Cref{fig:uu-init } shows a tomography of carbon-1 and carbon-4 initialized in the $\ket{00}$-state.
In the ideal case the single qubit Z-measurements and the ZZ parity are 1 and all other coefficients are 0.
The ideal case is represented by the gray bars in the \cref{fig:uu-init }.
The fidelity to the ideal case is $F = 81.43 \pm 1.68$ \%.

\Cref{fig:ud-init } shows a tomography of carbon-1 and carbon-4 initialized in the $\ket{01}$-state.
In the ideal case ZI =1 and IZ and ZZ are -1, all other coefficients are 0.
The ideal case is represented by the gray bars in the \cref{fig:ud-init }.
The fidelity to the ideal case prediction is $80.99 \pm 1.69$ \%.

\begin{figure}[htbp]
    \begin{subfigure}[t]{0.49\textwidth}\centering
        \caption{}
        \includegraphics{Img/uu-no-parity.pdf}
        \label{fig:uu-init }
    \end{subfigure}
    \begin{subfigure}[t]{0.49\textwidth}\centering
        \caption{}
        \includegraphics{Img/ud-no-parity.pdf}
        \label{fig:ud-init }
    \end{subfigure}
    \caption{ Quantum state tomographies of two initialized carbons. In \cref{fig:uu-init } the carbons are initialized into $\ket{00}$ with a fidelity of  $81.43 \pm 1.68$ \%.
    In \cref{fig:ud-init } the carbons are initialized into $\ket{10}$ with a fidelity of $80.99 \pm 1.69$ \%.
    }
    \label{fig:2qubitTomos}
\end{figure}

\section{Demonstrating entanglement between weakly coupled carbons}
Now that we are able to initialize multiple weakly coupled carbon spins and verify this with a tomography it is possible to demonstrate entanglement.
After the qubits are initialized an XX-parity is performed and the tomography is conditionalized on the negative readout result.

By conditionalizing on the negative readout result the carbons are projected into the negative-parity eigenstates of the XX-parity operator.
These are $\ket{-X,X}$ and $\ket{-X,X}$.
The tomography coefficient for the XX-parity is trivially -1.
The other coefficients depend on the initial state the parity was performed on.
The expected coefficients can be found as the expectation values of the operators on the state: $\bra{\Psi} \sigma_i \otimes \sigma_j \ket{\Psi}$ and are again depicted as gray bars.

\Cref{fig:uu-XX} shows the tomography for the negative XX parity of the $\ket{00}$-state.
$\ket{00}$ can be written in the X-basis as:
\begin{equation}
      \tfrac{1}{2} \left( \ket{X,X} + \ket{X,-X} +\ket{X,-X} + \ket{-X,-X} \right)
 \end{equation}
By measuring the negative XX-parity the state is projected onto:
\begin{equation}
    \tfrac{1}{\sqrt{2}} \left( \ket{X,-X} +\ket{X,-X} \right)
\end{equation}
The expectation for the ideal case is represented by the gray bars.
The fidelity to the ideal case prediction is $76.60 \pm 1.74$ \%.

\paragraph{ }
\Cref{fig:ud-XX} shows the tomography for the negative XX parity of the $\ket{01}$-state.
$\ket{01}$ can be written in the X-basis as:
\begin{equation}
    \tfrac{1}{2} \left( \ket{X,X} - \ket{X,-X} +\ket{X,-X} - \ket{-X,-X} \right)
 \end{equation}
By measuring the negative XX-parity the state is projected onto:
\begin{equation}
    \tfrac{1}{\sqrt{2}} \left( \ket{X,-X} -\ket{X,-X} \right)
\end{equation}
The expectation for the ideal case is represented by the gray bars.
The fidelity to the ideal case prediction is  $76.08 \pm 1.74$ \%


\begin{figure}[htbp]
    \begin{subfigure}[t]{0.49\textwidth}\centering
        \caption{}
        \includegraphics{Img/uu-XX-parity.pdf}
        \label{fig:uu-XX}
    \end{subfigure}
    \begin{subfigure}[t]{0.49\textwidth}\centering
        \caption{}
        \includegraphics{Img/ud-XX-parity.pdf}
        \label{fig:ud-XX}
    \end{subfigure}
    \caption{ Quantum state tomographies for entangled states created by measuring a negative XX parity.
    \Cref{fig:uu-XX} is the negative XX-parity of the state prepared in \cref{fig:uu-init }. It is in the $    \tfrac{1}{\sqrt{2}} \left( \ket{X,-X} +\ket{X,-X} \right)
$ with  $76.60 \pm 1.74\%$ fidelity.
    \Cref{fig:ud-XX} is the negative XX-parity of the state prepared in \cref{fig:ud-init }. It is in the $\tfrac{1}{\sqrt{2}} \left( \ket{X,-X} -\ket{X,-X} \right)$  with  $76.08 \pm 1.74$ \% fidelity.
    }
    \label{fig:2qubit_parity_Tomos}
\end{figure}

Some conclusion...

