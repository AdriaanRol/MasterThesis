\chapter{Controlling Weakly-coupled Carbon Spins}
The creation of entanglement is an essential capability for QEC in particular and quantum-computation in general.
This chapter demonstrates how weakly coupled nuclear spins can be initialized and read-out and how this can be used to generate entanglement between them.

\section{Initialization and readout of single spins}
\label{sec:carbon_init_and_readout}
A weakly coupled spin can be initialized by conditionalizing on a readout result, similar to how nitrogen-MBI works.
A weakly coupled carbon spin can be read-out by using the $\pm \mathrm{x}$-gate to entangle the phase of the electronic-spin with the state of the nuclear spin and reading out the phase of the electronic spin.

%Explain Readout.
The gates used to initialize a weakly coupled nuclear spin are depicted in \cref{fig:gate_circuit_initialization}.
The circuit depicted in \cref{fig:gate_circuit_mbi_x-init} is used to perform a read-out along the $x$-axis.
By applying a $\pm y$-gate on the carbon before the initial $y$-pulse the state can be read out along the $z$-axis.

%Explain Initialization.
\begin{figure}[htbp]
    \centering
    \begin{subfigure}[t]{0.49\textwidth}
    \centering
    \caption{}
    \mbox{
        \Qcircuit @C=1em @R=.7em {
        \lstick{\ket{0}_e}                        & \gate{\mathrm{y}}  & \ctrl{1}       & \gate{\mathrm{x}} &\qw          &  \meter \\
        \lstick{\rho_\mathrm{m}}         & \qw              &  \gate{\pm \mathrm{x}}     & \qw    & \qw   & \qw}}
    \label{fig:gate_circuit_mbi_x-init}
    \end{subfigure}
    \begin{subfigure}[t]{0.49\textwidth}
        \centering
        \caption{}
        \mbox{
        \Qcircuit @C=1em @R=.7em {
            \lstick{\ket{0}_e} & \gate{\mathrm{y}}  & \ctrl{1} & \gate{\mathrm{x}} &\ctrl{1} &  \meter \\
            \lstick{\rho_\mathrm{m}}& \qw&  \gate{\pm \mathrm{x}}     & \qw    & \gate{\mp \mathrm{y}}    & \qw}}
        \label{fig:gate_circuit_mbi_swap-init}
    \end{subfigure}
    \caption{\Cref{fig:gate_circuit_mbi_x-init} MBI-based initialization into $\pm \ket{\mathrm{X}}$. Initializes the carbon into $\ket{X}_\mathrm{C} $ when $\ket{0}_e$ is measured and into $\ket{-X}_\mathrm{C} $ when $\ket{1}_e$ is measured for the electron.
    \Cref{fig:gate_circuit_mbi_swap-init} MBI-swap initialization into $ \ket{\mathrm{0}}$. Initializes the carbon into $\ket{0}_\mathrm{C} $ regardless of the electronic spin-state measured.}
    \label{fig:gate_circuit_initialization}
\end{figure}


% To initialize the nuclear spin
The initial pulse in \cref{fig:gate_circuit_mbi_x-init} brings the electronic spin in $\ket{X}_e$.
When the nuclear spin starts in the mixed state the system can be described by the tensor product of two density matrices:
\begin{equation}
    \rho_X \otimes \rho_m = \rho_X \otimes \rho_{X} +\rho_X \otimes \rho_{-X}
\end{equation}
By applying the $\pm{\mathrm{x}}$-gate  the electronic-spin picks up a phase depending on the nuclear spin-state:
\begin{equation}
     \rho_Y \otimes \rho_{X} +\rho_{-Y} \otimes \rho_{-X}
\end{equation}
By reading out the electronic spin along the $y$-axis the nuclear spin is projected into the $\ket{X}_C$ or $\ket{-X}_C$-state.
By conditionalizing on a positive readout result the state can be initialized into the $\ket{X}_C$-state.

By adding a second conditional gate (\cref{fig:gate_circuit_mbi_swap-init}) the nuclear spin is brought into the $\ket{0}_C$-state regardless of the measurement outcome for the electronic spin.

\paragraph{}
\Cref{fig:single_qubit_initialization} demonstrates initialization and readout.
In \cref{fig:carbon_init_x} carbon-1 is initialized into the $\ket{X}_C$-state and in \cref{fig:carbon_init_Z} it is initialized into the $\ket{0}_C$-state.
This is done by implementing the circuits depicted in \cref{fig:gate_circuit_initialization} and conditionalizing on a positive electron readout.

The carbon is read out using the same circuits.
The blue points correspond to $x$-readout, the green points to $y$-readout and the red-points to the $z$-readout.
The phase is sweeped to demonstrate that the readouts function as intended.
For a qubit in the $\ket{0}$-state the initial phase is undefined.

The combined fidelity of readout and initialization to the desired state is: $F(\ket{X}) = 90.57 \pm 0.85 \% $ for the $\ket{X}$-state and $F(\ket{Z}) = 93.00 \pm 0.31 \%$ for the $\ket{Z}$-state.
% Add description of features if still have time after finished today :)

\begin{figure}[htbp]
    \begin{subfigure}[t]{0.49\textwidth}\centering
        \caption{}
        \includegraphics{Img/RO_and_init_C1_X.pdf}
        \label{fig:carbon_init_x}
    \end{subfigure}
        \begin{subfigure}[t]{0.49\textwidth}\centering
        \caption{}
        \includegraphics{Img/RO_and_init_C1_Z.pdf}
        \label{fig:carbon_init_Z}
    \end{subfigure}
    \caption{Demonstration of carbon initialization and readout. In \cref{fig:carbon_init_x} carbon-1 is initialized in $\ket{X}$ and read-out. In \cref{fig:carbon_init_Z} carbon-1 is initialized in $\ket{0}$. Colored points correspond to readouts in different bases, blue to $x$-readout, green to $y$-readout and red to $z$-readout.}
    \label{fig:single_qubit_initialization}
\end{figure}


\section{Parity measurement}
Entanglement can be created by performing a parity measurement.

A parity measurement does not measure the state of two qubits but measures if  two qubits are the same in a certain basis.
An example is the XX-parity measurement.
The XX-parity measurement returns a positive result if the the two qubits are the same in the $x$-basis and a negative result if the two are opposite.
That is it returns a positive result if the state is $\ket{X,X}$ or $\ket{-X,-X}$ and a negative result if the state is $\ket{X,-X}$ or $\ket{-X,X}$.
In general a two qubit parity operator has 2 eigenvalues, both are twofold degenerate.

The XX-parity measurement can be implemented on a weakly coupled spin using the circuit depicted in \cref{fig:gate_circuit_general_Parity_RO}.
A parity measurement is very similar to the regular readout depicted in \cref{fig:gate_circuit_mbi_x-init}.
Once the electron is brought into a superposition the electron picks up phase when the $\mp \mathrm{x}$-gate is applied.
$+\pi/2$-phase when the carbon is in $\ket{+X}$ and $-\pi/2$-phase when the carbon is in $\ket{-X}$.
This is done for both carbons, when both carbons are in the same $x$-state the electron will pick up $\pi$-phase but when they do not give the same result the phase cancels.
By reading out the electronic-spin along $x$ the parity is measured.
It should be noted that because we use a $\pm \mathrm{x}$-gate instead of a CNOT-gate an additional $\pi/2$-phase is added to the carbon states compared to a regular parity measurement.

\begin{figure}[htbp]
    \centering
\mbox{
\Qcircuit @C=1em @R=.7em {
\lstick{\ket{0}_e} &  \gate{\mathrm{y}}  & \ctrl{1} &  \ctrl{2} & \gate{\mathrm{y}}  &  \meter &\qw\\
\lstick{\ket{\psi}_{C1}} &  \qw & \gate{\mp \mathrm{x}}  &\qw   &  \qw   &\qw&\qw \\
\lstick{\ket{\psi}_{C2}}   & \qw   & \qw    & \gate{\mp \mathrm{x}}   &\qw & \qw &\qw}}
    \caption{Gate circuit for a XX-parity measurement. }
    \label{fig:gate_circuit_general_Parity_RO}
\end{figure}

\section{Demonstrating entanglement}
To demonstrate entanglement it must be verified that the entangled state is created.
This can be done by performing a state tomography.
In quantum state tomography the density matrix of a quantum state is reconstructed by repeatedly preparing the same state and gathering measurement statistics.

An arbitrary matrix can be described as a weighted sum of the Pauli-matrices and the Identity as in: \cref{eq:pauli}.
\begin{equation}
    \rho = I + \sum_{i,j} a_{i,j} \sigma_i \otimes \sigma_j
    \label{eq:pauli}
\end{equation}
By measuring the coefficients of \cref{eq:pauli}  the density matrix $\rho$ can be reconstructed completely.


\subsection{Readout}
To measure the Pauli matrix single and multi qubit measurements are needed.
Single qubit measurements were described in \cref{sec:carbon_init_and_readout}.
The two qubit measurements that are required are similar to parity measurements.
These measurements have to measure the parity but it is not important whether the measurement preserves the state.

\Cref{fig:gate_circuit_general_RO} shows a schematic representation of a general two-qubit readout that determines the parity.
By choosing the phase of gate $c$ and $d$ the parity with respect to X or Y can be read out.
By adding the dotted gates $a$ or $b$ the parity with respect to Z can be read out. When these gates are added however the phase difference between $a$ and $c$, and $b$ and $d$ must be ensured to be $90^\circ$.
As an example to measure the YZ , gate $a$ is not implemented, gate $b$ is implemented, gate $c$ is $\pm \mathrm{y}$ and gate $d$ is also $\pm \mathrm{y}$

\begin{figure}[htbp]
    \centering
\mbox{
\Qcircuit @C=1em @R=.7em {
\lstick{\ket{0}_e} &  \dashedCtrl{1} &   \dashedCtrl{2} &  \gate{\mathrm{-y}}  &  \ctrl{1} &   \ctrl{2} & \gate{\mathrm{y}}  &  \meter &\qw\\
\lstick{\ket{\psi}_{C1}} &  \dashedGate{\pm \mathrm{x}_a}   &\qw &  \qw & \gate{\pm \mathrm{\pi/2}_c}  &\qw   &  \qw   &\qw&\qw \\
\lstick{\ket{\psi}_{C2}}  & \qw    & \dashedGate{\pm \mathrm{x}_b}  & \qw   & \qw    & \gate{\pm \mathrm{\pi/2}_d}   &\qw & \qw &\qw}}
    \caption{Gate circuit for a general two qubit readout. }
    \label{fig:gate_circuit_general_RO}
\end{figure}

% How readout is implemented
\subsection{Expectation Initialized states? }
To create entanglement we first verify that we have correctly initialized our qubits.
To demonstrate entanglement we first verify that our tomography is working correctly on a simple initalized state.
We then perform a parity measurement on these states and compare that to the ideal case.

\Cref{fig:uu-init } shows a tomography of carbon-1 and carbon-4 initialized in the $\ket{00}$-state.
For this state we expect in the ideal case the single qubit Z-measurements (IZ and ZI) and the ZZ parity to be 1 and all others to be 0.
The ideal case is represented by the gray bars in the tomography.
The fidelity to ideal case prediction is $F = 81.43 \pm 1.68$ \%.

\Cref{fig:ud-init } shows a tomography of carbon-1 and carbon-4 initialized in the $\ket{01}$-state.
For this state we expect in the ideal case ZI =1, IZ = -1 and ZZ = -1. All other measurements should give 0 contrast.
The ideal case is represented by the gray bars in the tomography.
The fidelity to the ideal case prediction is $80.99 \pm 1.69$ \%.

\paragraph{ }
After initializing the qubits we perform an XX-parity measurement on the qubits and conditionalize on the negative readout result.
This projects the state into the negative parity eigenstates of the XX-parity.
These are $\ket{-X,X}$ and $\ket{-X,X}$.
The result of the XX-parity in the tomography is trivially -1. The other coefficients however depend on the initial state the parity was performed on.

\Cref{fig:uu-XX} shows the tomography for the negative XX parity of the $\ket{00}$-state.
$\ket{00}$ can be written in the X-basis as:
\begin{equation}
      \tfrac{1}{2} \left( \ket{X,X} + \ket{X,-X} +\ket{X,-X} + \ket{-X,-X} \right)
 \end{equation}
By measuring the negative XX-parity the state is projected onto:
\begin{equation}
    \tfrac{1}{\sqrt{2}} \left( \ket{X,-X} +\ket{X,-X} \right)
\end{equation}
The fidelity to the ideal case prediction is $76.60 \pm 1.74$ \%.


\Cref{fig:ud-XX} shows the tomography for the negative XX parity of the $\ket{01}$-state.
$\ket{01}$ can be written in the X-basis as:
\begin{equation}
    \tfrac{1}{2} \left( \ket{X,X} - \ket{X,-X} +\ket{X,-X} - \ket{-X,-X} \right)
 \end{equation}
By measuring the negative XX-parity the state is projected onto:
\begin{equation}
    \tfrac{1}{\sqrt{2}} \left( -\ket{X,-X} +\ket{X,-X} \right)
\end{equation}
This obviously
The fidelity to the ideal case prediction is  $76.08 \pm 1.74$ \%

% Results of two qubit initialization uu + expectation.
\subsection{Parity in pauli}
%Expectation for



%expected result /gray bars
Init in state uu
projected into -XX
-> state k
in pauli basis looks like ..


\begin{figure}[htbp]
    \begin{subfigure}[t]{0.49\textwidth}\centering
        \caption{}
        \includegraphics{Img/uu-no-parity.pdf}
        \label{fig:uu-init }
    \end{subfigure}
    \begin{subfigure}[t]{0.49\textwidth}\centering
        \caption{}
        \includegraphics{Img/ud-no-parity.pdf}
        \label{fig:ud-init }
    \end{subfigure}

    \begin{subfigure}[t]{0.49\textwidth}\centering
        \caption{}
        \includegraphics{Img/uu-XX-parity.pdf}
        \label{fig:uu-XX}
    \end{subfigure}
    \begin{subfigure}[t]{0.49\textwidth}\centering
        \caption{}
        \includegraphics{Img/ud-XX-parity.pdf}
        \label{fig:ud-XX}
    \end{subfigure}
    \caption{ Preliminary data
    \cref{fig:uu-init } Fidelity to theoretical prediction  = $81.43 \pm 1.68$ \% \\
    \cref{fig:ud-init } Fidelity to theoretical prediction  = $80.99 \pm 1.69$ \% \\
    \cref{fig:uu-XX}  Fidelity to theoretical prediction  = $76.60 \pm 1.74$ \% \\
    \cref{fig:ud-XX}  Fidelity to theoretical prediction  = $76.08 \pm 1.74$ \%
    }
    \label{fig:2qubitTomos}
\end{figure}

% What bars to include?
% Experiment that initializes both carbons + bar graph? -> to explain how to read + to explain

% 20140810 122740 Z Z init
%  20140810 132412  YYparity
%
% \begin{figure}[htbp]
%     \centering
% \mbox{
% \Qcircuit @C=1em @R=.7em {
% \lstick{\ket{0}_e} & \gate{\mathrm{y}}  & \ctrl{1} &  \ctrl{2} & \gate{y}  &  \meter &\qw\\
% \lstick{\ket{0}_{C1}} & \qw&  \gate{\pm \mathrm{x}}  &\qw  & \qw       &\qw&\qw& \\
% \lstick{\ket{0}_{C2}}& \qw& \qw  & \gate{\pm \mathrm{x}}    & \qw      &\qw&\qw&}}
%     \caption{XX-parity-measurement}
%     \label{fig:gate_circuit_XX-parity-measurement}
% \end{figure}



%  Should emphasize difficulty in seperating initialization and RO fidelity, what is not working? Is it working?


