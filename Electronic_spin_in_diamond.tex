\chapter{Electronic spins in Diamond}
\label{controlingspinsindiamond}
%TODO_MAR:  Status: Up for revision
% TODO_MAR: Fix definition of weakly coupled carbons


% TODO_MAR: Rewrite introduction of diamond spin control chapter

%Explain here how general experiment looks

It has been shown that the nuclear- and electron- spin-state of the NV- center can be initialized, coherently controlled and read-out using microwave- and laser- pulses\cite{Robledo2011HighFidelity}. In these experiments two lasers that are resonant with transitions in the NV- center are used to initialize the electronic spin state. One of these two lasers is used to read out the electronic spin state and an off-resonant laser is used to reset the system. \comment{TODO_MAR: Check: Was the off-resonant laser already used in the Robledo paper?} Microwaves are used to drive transitions between the different nuclear and electronic spin states.

Strongly coupled nuclear spins can be initialized by conditionally rotating the electronic state to a state that is read out only if the Carbon is in the desired state, when the electronic state readout has a positive result the system is projected into the desired state. We call this Measurement Based Initialization (MBI).

Our experiments are build around the same basic tools\comment{As the Robledo experiment? }. Each experiment starts with a Charge-Resonance check \comment{Explain how CR-Check works? Ref?}that verifies if the lasers are still on resonance. After that the Nitrogen state is initialized using MBI. Once the system is initialized our actual experiment is performed. This consists of microwave pulses and a single or multiple readouts.

All experiments were performed on a custom-build cryostat setup operating at liquid helium temperatures described in detail in chapter 3 of \cite{Bernien2014Control}. It was additionally outfitted with a movable magnet that applied a magnetic field of ~300G to the sample.

\section{Spin Control}
\label{spincontrol}

The electronic ground state Hamiltonian can be written as\cite{Pfaff2013Quantum}:
 \begin{equation}
H_{GS} = \Delta {S_z}^2 + \gamma_e \mathbf{B} \cdot \mathbf{S}
\end{equation}

With zero field splitting $\Delta \approx 2.88 \mathrm{GHz}$  and gyro-magnetic ratio $\gamma_e  = 2.802$ MHz/G . In this expression the interactions with the Nitrogen nucleus and the Carbon spin bath are not included. By applying a magnetic field $B_z$ a long the NV axis the degeneracy of the  $m_s =\pm1$ states is lifted by the Zeeman effect. A two level system that serves as a qubit can be defined with  $m_s=0:=|0\rangle$ and $m_s = -1 := |1\rangle$.

On the Bloch-sphere the state vector rotates around the quantisation axis with a frequency depending on the energy splitting of the two states given by the Larmor frequency  $\omega_L =\Delta - \gamma_e {B_z} $.\footnote{When  $\omega_L$  is used as a vector it is pointing in the $\hat{z}$ direction.} By applying an external field a term is effectively added to the Hamiltonian, changing the quantization axis and thereby its evolution. By applying microwaves with the right frequency this can be used to selectively drive the transition from the  $|0\rangle$ state to the $|1\rangle$state\cite{Jelezko2004Observation}.


\section{Structure of a typical experiment}
% Explain pump laser and excitation laser,
% Explain experiment starts with checking if Lasers are on resonance
% Explain Nitrogen initialization to remove term from the Hamiltonian

% Explain structure of typical experiment and experimental setup
% Setup, same as in old papers. State temperature 4K, state that magnetic field is used.
%Structure.
%CR-check , Init electron with lasers, Init Nitrogen, do fancy pulse sequence, readout, possibly feed forward, more pulses, RO again.



% \section{Dynamical decoupling}
% DONT need to explain general dynamical decoupling?
%Explain what dynamical decoupling is, decoupling from environment by 'inverting' the enviroment thus reducing influence of decoherence causing carbons.
%Show figure from Tim's paper showing extended electron coherence times.
% Make bridge to next chapter by refering paper again showing that it's possible to control carbons in this way by resonantly doing this
