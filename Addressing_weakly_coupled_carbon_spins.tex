
\chapter{Addressing Weakly-coupled Carbon Spins}

In order to be less dependent on the presence of strongly coupled nuclear spins weakly coupled spins can be addressed.
This chapter will demonstrate how to extend electron coherence time so that weakly coupled spins can be resolved.
Identify nuclear spins in the environment of our NV-center and characterize these spins.







The electronic-spin of the NV-center does not live in a vacuum but in an environment full of nuclear spins.
To some of these spins the NV-center couples strongly, these spins can be controlled and can serve as qubits.
To others it couples weakly these spins are a source of decoherence and cannot be controlled directly.

In this chapter I will first explain what strong and weakly coupled spins are, how this relates to coherence and how coherence can be extended trough dynamical decoupling.
In the second part of this chapter I will explain how dynamical dynamical decoupling can be used to identify and control some of these weakly coupled spins, transforming them from a source of decoherence to a resource for qubits.

\begin{figure}[htbp]
\centering
    \begin{tikzpicture}
        \node[anchor=south west,inner sep=0] at (0,0) {\includegraphics[scale = 0.5]{Img/coupling_regimes.eps}};
        \node[font=\tiny] at (5,4) {$A \approx \frac{\sqrt{\ln{2}}}{\pi T_{2\mathrm{C}}^*}$};
        \node[font=\tiny] at (4,3.1) {$A\approx \frac{\sqrt{\ln{2}}}{\pi T_{2e}^*}$};
        \node[font=\small] at (3.8,1.9) {$\mathbb{I}$};
        \node[font=\small] at (4.75,1.2) {$\mathbb{II}$};
        \node[font=\small] at (5.6,0.5) {$\mathbb{III}$};
        %Help grid for drawing
        % \draw[help lines,xstep=1,ystep=1] (0,0) grid (7,5);
        % \foreach \x in {0,1,...,7} { \node [anchor=north] at (\x,0) {\x}; }
        % \foreach \y in {0,1,...,5} { \node [anchor=east] at (0,\y) {\y}; }
    \end{tikzpicture}
    \caption{ Schematic representation of different coupling regimes. In the strong coupling regime (region $\mathbb{I}$) carbon-spins are coupled to the NV-center stronger than the coupling of the NV-center to the spin-bath. These carbons can be addressed directly. In the weak coupling regime (region $\mathbb{II}$) carbon-spins are coupled more strongly to the NV-center than to the spin-bath but not strong enough to be addressed directly. In the very-weak coupling regime (region $\mathbb{III}$) the coupling to the spin-bath is stronger than the coupling to the NV-center. These spins cannot be addressed.}
    \label{fig:coupling regimes}
\end{figure}
