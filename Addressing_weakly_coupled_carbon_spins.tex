
\chapter{Addressing Weakly-coupled Carbon Spins}
The spin register can be extended by addressing weakly-coupled spins.
This chapter will explain how coherence can be extended trough dynamical decoupling and how this can be used to address single carbon-spins.
The methods will be used to identify and characterize individual nuclear spins.

[NOTE: Maybe remove following paragraph and figure?]

\Cref{fig:coupling regimes} shows a schematic representation of different coupling regimes.
Carbons in region $\mathbb{I} $ are in the strong coupling regime, transitions of these spins can be readily resolved and they can be controlled using the methods described in the previous chapter.
By extending the coherence of the electronic spin certain weakly coupled spins can be controlled \citep{Taminiau2012Detection}.
These can be found in region $\mathbb{II}$
Control of spins in region $\mathbb{III}$ is not limited by the coherence of the electronic-spin but by that of the nucleus.
These spins cannot be addressed without inventing new methods that significantly improve both electronic- and nuclear- spin coherence.


\begin{figure}[htbp]
\centering
    \begin{tikzpicture}
        \node[anchor=south west,inner sep=0] at (0,0) {\includegraphics[scale = 0.5]{Img/coupling_regimes.eps}};
        \node[font=\tiny] at (5,4) {$A \approx \frac{\sqrt{\ln{2}}}{\pi T_{2\mathrm{C}}^*}$};
        \node[font=\tiny] at (4,3.1) {$A\approx \frac{\sqrt{\ln{2}}}{\pi T_{2e}^*}$};
        \node[font=\small] at (3.8,1.9) {$\mathbb{I}$};
        \node[font=\small] at (4.75,1.2) {$\mathbb{II}$};
        \node[font=\small] at (5.6,0.5) {$\mathbb{III}$};
        %Help grid for drawing
        % \draw[help lines,xstep=1,ystep=1] (0,0) grid (7,5);
        % \foreach \x in {0,1,...,7} { \node [anchor=north] at (\x,0) {\x}; }
        % \foreach \y in {0,1,...,5} { \node [anchor=east] at (0,\y) {\y}; }
    \end{tikzpicture}
    \caption{ Schematic representation of different coupling regimes. In the strong coupling regime (region $\mathbb{I}$) carbon-spins are coupled to the NV-center stronger than the coupling of the NV-center to the spin-bath. These carbons can be addressed directly. In the weak coupling regime (region $\mathbb{II}$) carbon-spins are coupled more strongly to the NV-center than to the spin-bath but not strong enough to be addressed directly. In the very-weak coupling regime (region $\mathbb{III}$) the coupling to the spin-bath is stronger than the coupling to the NV-center. These spins cannot be addressed.}
    \label{fig:coupling regimes}
\end{figure}
\section{Extending Electron Coherence}

To be able to resolve weakly coupled carbons it is necessary to extend the coherence of the electron spin.
By using a spin-echo the effect of variations in the environment \emph{between} experiments can be eliminated, making variations of the spin-bath \emph{during} a single sequence (between $\pi/2$ pulses) the main source of decoherence.
By dynamical decoupling the effect of these nuclear spin-spin dynamics on the coherence can be minimized and the interactions with the spin-bath can be exposed.


\subsection{Spin-Echo}

A spin echo experiment (\cref{fig:spin_echo_gijs}) is very similar to a Ramsey experiment.
The difference is an additional $\pi$ pulse that is added in the middle of the experiment exactly between the $\pi/2$ pulses of the Ramsey sequence.
In a spin echo the state is brought into the $xy$-plane where it evolves for a time $\tau/2$ before a $\pi$-pulse, along the y-direction in the rotating frame, is applied.
It evolves for another $\tau/2$ before a final $\pi/2$-pulse ideally rotates it back towards $\ket{0}$ and it is read out.

The key component of a spin-echo is the central $\pi$-pulse that cancels out the effect of quasi-static variations in the spin-bath configuration.
The $\pi$-pulse can be seen as turning the reference frame of the NV-spin upside down.
If the spin-bath configuration is approximately static during the sequence, the detuning of the evolution frequency with respect to the central frequency during the first part will be exactly opposite to the detuning during the second part.
This means that any phase difference picked up during the first half of the evolution is canceled out during the second half of the evolution.
\begin{figure}[htbp]
    \centering
    \includegraphics{Img/SpinEcho_Gijs.pdf}
    \caption{In a spin echo experiment the qubit is brought into the $xy$-plane of the Bloch-sphere by a $\pi/2$-pulse. Here it freely evolves for a time $\tau/2$ before being flipped by a $\pi$-pulse along the $y$-axis of the rotating frame. It is let to evolve for another $\tau/2$ before a final $\pi/2$ pulse brings is used to read out the $x$-component.
    If the spin-bath configuration does not change during the free evolution time $\tau$ the state vector will end up along the $x$-axis irrespective of the initial spin-bath configuration.
    Figure from \citet{Lange2012Quantum}. }
    \label{fig:spin_echo_gijs}
\end{figure}


Due to nuclear spin-spin interactions the spin-bath does not remain static during experiments and the cancellation is not perfect, some phase is picked up causing the signal to ultimately decohere.
This coherence time ($T_2$) is defined as the $1/e$ value of the decay of a spin echo experiment and measures decoherence due to dynamics in the spin-bath during an experiment.
$T_2$ was measured to be $1.10 \pm 0.01\, \mathrm{ms}$.
% Data from exp:  20140405/123712

\subsection{Dynamical Decoupling}
A natural way to extend the phase cancellation properties of the spin-echo experiment to shorter timescales is by applying more $\pi$-pulses.
This procedure is known as dynamical decoupling.
Similar to how the spin-echo cancels out phase picked up due to any variations that are quasi-static on the time-scale of the experiment, dynamical decoupling cancels out phase due to variations on the time-scale of the $\pi$-pulses.
Dynamical decoupling can significantly improve coherence times \citep{Lange2010Universal}.

On the NV-center used in this thesis a coherent signal\footnote{$F\ket{0} > 0.68$} is measured after more than $40 \,\mathrm{ms}$ for 256 pulses.
Work on ensembles indicates that that this can be improved even further by applying more pulses: a coherence time of $T_{DD} \approx 0.6 \, \mathrm{s}$ was reported at $77\, \mathrm{K}$ \citep{Gill2013SolidState}.


% Although dynamical decoupling improves the coherence of the central spin by decoupling from the environment, the central spin is also decoupled from other spins preventing direct two-qubit gates. It was demonstrated by \citet{Sar2012DecoherenceProtected} how to incorporate dynamical decoupling in a universal gate design by implementing Grover's algorithm.
% Using this technique \citet{Taminiau2012Detection} used the extended coherence to detect and control weakly-coupled carbon spins, before implementing three-qubit quantum-error-correction (QEC) \citep{Taminiau2014Universal}.

% As these experiments where performed with NV-centers at Room temperature they lack the option to do single-shot readout required to act on a measurement outcome\footnote{@Tim, I think this can be worded more concisely. Do you have any ideas?}. An essential feature for the parity measurements that form the basis of measurement-based QEC and surface codes.

% As we cannot perform an ESR experiment while decoupling a different technique must be used to resolve and address additional spins.
% In order to resolve additional spins we perform a dynamical decoupling spectroscopy, resulting in a fingerprint of the nuclear-spin environment\citep{Taminiau2012Detection}.

\section{Identifying weakly-coupled carbon-pins}
To be able to control nuclear spins the dynamics must be understood and suitable spins must be identified.
This section will discuss the effect dynamical decoupling has on nuclear spins.
This knowledge is then used to explain the features in the fingerprint \cref{fig:FP} and identify several nuclear spins.

\subsection{The effect of dynamical decoupling}

\begin{figure}[htbp]
\centering

        \begin{tikzpicture}
            \node[anchor=south west,inner sep=0] at (0,0) {\includegraphics[keepaspectratio,width=0.15\textwidth]{./img/QuantizationAxis.png}
};
            \node[font=\small, text = blue] at (4.05,1.7)  {1};
        \end{tikzpicture}


\includegraphics[keepaspectratio,width=0.15\textwidth]{./img/QuantizationAxis.png}
\caption{Flipping the electron spin from the  $m_s=0$ to the $m_s= +1$ state changes the quantization axis of nuclear spins. For  $m_s=0$ all nuclear spins precess about $\bm{\omega_L}$. For  $m_s=+1$ each spin precesses about a distinct axis $\bm{\tilde{\omega}}=\bm{\omega_L} +\bm{A}$.}
\label{fig:quantax}
\end{figure}

When the electron is in the $m_s=0$ state each nuclear spin precesses about $\bm{\omega_L}$ with the Larmor frequency. When the electron is in the $m_s=+1$ state nuclear spins precess about a distinct axis $\bm{\tilde{\omega}}=\bm{\omega_L} +\bm{A}$ \citep{Taminiau2012Detection}. The hyperfine interaction $\bm{A}$ depends on the position of that particular nuclear spin relative to the NV- center.

When applying a decoupling sequence with N\slash 2 decoupling units of the form {$\tau - \pi -2\tau-\pi-\tau$}, the nuclear spin alternately rotates around the  $\bm\omega_L$ and the $\bm{\tilde{\omega}}$ axis.
The net result of one such decoupling sequence is a rotation around an axis $\bm{\hat{\mathrm{n_i}}}$ by an angle $\phi$.
Where $\bm{\hat{\mathrm{n_i}}}$ depends on the initial state of the electron: $\bm{\hat{\mathrm{n_0}}}$ when the electron starts in $m_s = 0$ and $\bm{\hat{\mathrm{n_1}}}$ when the electron starts in $m_s = +1$~\citep{Taminiau2012Detection}.

\begin{figure}[htbp]
    \begin{subfigure}[t]{0.49\textwidth}\centering
        \centering
        \caption{}
        \includegraphics{Img/unCond_rot_taminiau.pdf}
        \label{fig:uncond_rot}
    \end{subfigure}
    \begin{subfigure}[t]{0.49\textwidth}\centering
        \centering
        \caption{}
        \includegraphics{Img/Cond_rot_taminiau.pdf}
        \label{fig:cond_rot}
    \end{subfigure}
    \caption{\Cref{fig:uncond_rot} When the net rotation axes $\bm{\hat{\mathrm{n_0}}}$ and $\bm{\hat{\mathrm{n_1}}}$ point in the same direction the carbon experiences an unconditional rotation and cannot be controlled. \Cref{fig:cond_rot} When the net rotation axes $\bm{\hat{\mathrm{n_0}}}$ and $\bm{\hat{\mathrm{n_1}}}$ are anti-parallel the carbon experiences a conditional rotation, either around +x or -x, and can be controlled.}
    \label{fig:conditional_and_unconditional_rotation}
\end{figure}


To understand how a carbon-13 atom can be controlled it is useful to consider three situations. In the first situation the $\bm{\omega_L}$ and $\bm{A}$ point in the same direction. In the second situation $\bm{\omega_L}$ and $\bm{A_\perp}$ are of comparable magnitude, resulting in a large angle between the quantization axes. In the last situation $|\bm{A}|$ is small compared to  $\bm{|\omega_L|}$ resulting in a small angle between the quantization axes.

When $\bm{\omega_L}$ and $\bm{A}$ point in the same direction, the net rotation axis is independent of the initial electron-state making it impossible to use the electron to control the carbon-13 atom using this decoupling sequence.

In the case where $\bm{\omega_L}$ and $\bm{A_\perp}$ are of comparable magnitude the net rotation axes $\bm{\hat{\mathrm{n_i}}}$ are strongly dependent on the initial electron-state for almost any $\tau$. This creates entanglement between the electron and this carbon for a wide range of inter pulse-delays $\tau$. For future reference we say that these weakly-coupled carbons are in the \emph{complex regime}.

When considering the case where the hyperfine interaction is much smaller than the Larmor frequency ($\omega_L \gg |\bm{A}|$), the net rotation axes  $\hat{\mathrm{n_0}}$ and $\hat{\mathrm{n_1}}$ are practically parallel and the nuclear spin undergoes an unconditional evolution.
Only when the inter-pulse delay is precisely resonant with the spin dynamics the axes are anti-parallel leading to a conditional rotation\citep{Taminiau2012Detection}.
The resonant condition is given by \cref{eq:res_dip_loc}, where $k$ is an integer and the FWHM of the Lorentzian-shaped resonance is given by \cref{eq:res_dip_width}.
To distinguish these carbons from those in the complex regime we say that these weakly-coupled carbons are in the \emph{basic regime}.

% somehow state that resonance gives a dip.
 \begin{equation}
\tau = \frac{(2k+1)\pi}{2 \omega_L + A_\parallel}
\label{eq:res_dip_loc}
\end{equation}
 \begin{equation}
\Delta = \frac{A_\perp}{2 \omega_L^2}
\label{eq:res_dip_width}
\end{equation}

If  $\hat{n_0}$ and $\hat{n_1}$ are not parallel, the resulting conditional rotation of the nuclear spin generally entangles the electron and nuclear spins.


\subsection{Response of a carbon-spin to dynamical decoupling spectroscopy}

As a result the electronic spin, starting out in $\ket{X}$, entangles with the nuclear spin for specific values of $\tau$ during a dynamical decoupling spectroscopy. %NOTE DOES NOT ENTANGLE NECCESARILY
When reading out the electronic spin along the $x$-axis this creates a dip in the signal.
The probability that the initial state is preserved is given by \cref{eq:contrast_to_probability}. Where the contrast $M_j$ for a single nuclear spin is given by \cref{eq:contrast_single_carbon_spin}\citep{Taminiau2012Detection}.

\begin{equation}
\label{eq:contrast_to_probability}
P_x = (M+1)/2
\end{equation}

\begin{equation}
\label{eq:contrast_single_carbon_spin}
M_j = 1-(1 - \hat{\bm{\mathrm{n_0}}} \cdot \hat{\bm{\mathrm{n_1}}}) \sin^2 \frac{N\phi}{2}
\end{equation}

%alpha = \tilde{\omega} \tau
%beta = (\omega_L \tau)
% mz = (\frac{ A_ \parallel + \omega_L }{ \tilde{ \omega}})
\begin{equation}
\label{eq:vec_term}
    1 - \hat{\bm{\mathrm{n_0}}} \cdot \hat{\bm{\mathrm{n_1}}} =  \frac{A_\perp ^2}{\tilde{\omega^2}} \frac{(1- \cos{(\tilde{\omega} \tau)})(1-\cos{(\omega_L \tau)})} {1 +\cos{(\tilde{\omega} \tau)}\cos{(\omega_L \tau)} - (\frac{ A_ \parallel + \omega_L }{ \tilde{ \omega}}) \sin{(\tilde{\omega} \tau)}\sin{(\omega_L \tau)}}
\end{equation}
\begin{equation}
\label{eq:angle_term}
    \phi =  \cos^{-1}\left(\cos(\tilde{\omega} \tau) \cos(\omega_L \tau)-\left(\frac{ A_ \parallel + \omega_L }{ \tilde{ \omega}}\right) \sin(\tilde{\omega} \tau)\sin(\omega_L \tau)\right)
\end{equation}
In reality the electron is not interacting with a single carbon but with a bath of carbon atoms. When the electron interacts with multiple carbons at the same time the contrast $M$ is given by the product of all individual values $M_j$ for each individual spin $j$ (\cref{eq:prod_multiple_spins}). In order to selectively control one carbon the electron should not entangle with any other carbon when addressing it.

\begin{equation}
\label{eq:prod_multiple_spins}
    M = \prod_{j}{M_j}
\end{equation}

When entanglement is created with multiple carbons at the same time coherence is quickly lost and contrast drops to 0.
By sweeping the number of pulses $\pi$-pules the response of an individual carbon can be distinguished from the response of multiple spins.
Only when an individual carbon is being addressed is it possible to sweep the contrast to -1.


\subsection{Identifying Individual Carbon-spins}

By identifying distinct dips in the fingerprint, of which \cref{fig:FP} shows a small part, we are able to make a first estimate of the hyperfine coupling using their location(\cref{eq:res_dip_loc}) and their width (\cref{eq:res_dip_width}.
We then compute the responses for these estimated hyperfine parameters using \cref{eq:contrast_single_carbon_spin}.
The parameters are varied until the computed response agrees with the data as well as possible to arrive at a more accurate estimation of the hyperfine parameters.
Using this method 13 distinct carbon spins where identified.

The parameters of the 4 strongest coupled carbons are listed in \cref{tbl:HF_par} and their computed responses are visible as colored lines in \cref{fig:FP}.
All estimated hyperfine parameters and a link to the full fingerprint measurements can be found in \cref{chap:Fingerprint_data_appendix}.

\begin{table}[htbp]
\centering
    \begin{tabular}{cllll}
    Carbon & \quad \quad  $A_{\parallel} $ & \quad \quad $A_{\perp}$ \\ \hline
    1         & $2 \pi \cdot${ }30.0 kHz             & $2 \pi \cdot${ }80.0 kHz                \\
    2         & $2 \pi \cdot${ }27.0 kHz             & $2 \pi \cdot${ }28.5 kHz              \\
    3         & $2 \pi \cdot$-51.0 kHz          & $2 \pi \cdot$105.0 kHz              \\
    4         & $2 \pi \cdot${ }45.1 kHz           & $2 \pi \cdot${ }20.0 kHz                \\
    \end{tabular}
    \caption{Estimated hyperfine parameters for spins 1 to 4 in \cref{fig:FP}.}
    \label{tbl:HF_par}
\end{table}



Most spins are relatively far away from the NV-center and have similar hyperfine couplings causing their resonances to overlap. This causes a broad feature with low coherence known as the spin-bath collapse. This feature is clearly visible in the fingerprint (\cref{fig:FP}) at $\tau/(4\tau_L) = m$ for odd $m$, where $\tau_L$ is the Larmor period ($\tau_L = \frac{2\pi}{\omega_L} $).

Spins that have a stronger than average hyperfine-interaction show up outside or at the edge of the spin-bath collapse.
Spins that are in the basic regime show up as a narrow dip.
Going to larger $\tau$ separates these dips further as the order of the resonance $k$ increases.
By looking at larger $\tau$ it is possible to resolve and address more resonances.
Several spins in the basic regime have been identified 3 of these are visible as colored lines in \cref{fig:FP}.
As computations are fundamentally limited by the coherence time there is a limit to the resonance-order that can be used to address carbons, making it impossible to resolve all weakly coupled spins.

Besides the carbons in the basic regime there are also weakly-coupled carbons that are more strongly coupled.
When a carbon in the complex regime is present in the NV-center this manifests itself as a resonance with strong oscillations on the side. Such a feature is also clearly visible in \cref{fig:FP}. We have identified the oscillations in the fingerprint as belong to a single spin which is denoted by the red line.

When a weakly coupled carbon in the complex regime is present a significant part of the fingerprint spectrum is inaccessible for controlling other carbons making them an undesired feature when attempting to control weakly coupled carbon spins.


\subsection{Effect of the magnetic field}

There are significant advantages to increasing the magnetic field when attempting to address weakly coupled carbons.
By increasing the magnetic field the Larmor frequency can be increased, reducing the number of carbons that are in the complex regime.
This causes the broad oscillating resonances to disappear allowing more carbons to be addressed.

Although increasing the magnetic field can improve the situation it is not always possible or desired.
When the magnetic field becomes too strong too strong the resonances become narrower than the resolution of the Arbitrary Waveform Generator used to generate the pulses that address the resonances, making it impossible to address these resonances effectively.
Simulations were performed (see \cref{chap:addressable_carbon_sims}) that indicate that for a natural carbon-13 concentration there is a range between 400G and 1400G where the magnetic field is optimal for controlling weakly coupled spins.

Besides the spin environment there are other factors affecting the choice for magnetic field.
Because the optical transitions used for readout and initialization depend on strain and magnetic field field\citep{Hensen2011MeasurementBased}, care must be taken when measuring that states do not mix in the excited state.
This combined with the fact that few experiments have been performed at high magnetic field and low temperature make it more practical to settle for a more moderate magnetic field of 300G.



\section{Characterizing weakly-coupled carbon spins}

This section will explain how a weakly-coupled spin can be controlled using the conditional rotation of the carbon spin that occurs when on resonance (\cref{eq:res_dip_loc}).
It will start by explaining how to do basic gate operations in the ideal case of being perfectly on resonant and not interacting with any nuclear spins.
After that we will explain how a carbon-spin in a mixed state can be initialized

\subsection{Basic operations}
As was explained in \cref{sec:controllingacarbonthroughdynamicaldecoupling} nuclear spins perform a rotation along two anti-parallel axes when subjected to a dynamical decoupling sequence on a resonant condition given by \cref{eq:res_dip_loc}.
The angle of rotation can be controlled by choosing the number of pulses of the decoupling sequence.

By choosing the number of pulses such that all coherence is lost when performing a dynamical decoupling spectroscopy measurement on the resonance, a rotation of $\pi/2$ in the clockwise direction is performed when the electron is in the $\ket{0}$-state and a $\pi/2$-rotation in the counterclockwise direction when the electron is in the $\ket{1}$-state.
We define the axis of rotation of this operation as the $x$-axis.

We call this conditional rotation the $\pm \mathrm{x}$-gate and it forms the basis of our control over weakly coupled spins. \Cref{fig:gate_circuit_pm-x} shows how we depict the $\pm \mathrm{x}$-gate in a circuit-diagram.
By letting the phase of the carbon evolve we are able to apply operations on the carbon-spin with arbitrary phase.
An unconditional gate can be implemented by placing the electron in an eigenstate before performing the $\pm\mathrm{x}$-operation.

\begin{figure}[htbp]
    \centering
        \mbox{
        \Qcircuit @C=1em @R=.7em {
         \lstick{\ket{\Psi}_e} &\ctrl{1}  &\qw\\
          \lstick{\ket{\Psi}_\mathrm{C}} &\gate{\pm \mathrm{x} }  &\qw}}
    \caption{The conditional x-gate ($\pm\mathrm{x}$). Performs an x-rotation on the carbon state ($\ket{\Psi }_\mathrm{C}$) when the electron is in the $\ket{0}_e$-state. It performs a $-\mathrm{x}$ rotation when the electron is in the $\ket{1}_e$-state.}
    \label{fig:gate_circuit_pm-x}
\end{figure}

Basic gates can be calibrated by sweeping the number of pulses $N$ when on resonance $\tau$.
In this manner carbon-1 and carbon-4 were found to perform the best $\pm\mathrm{x}$-gates.
The parameters used to implement $\pm\mathrm{x}$-gates are listed in \cref{tbl:gate_parameters}.

\begin{table}[htbp]
    \centering
    \begin{tabular}{cccc}
    Carbon &  $ N $ &  $\tau$ & total gate time\\ \hline
    1 &  18 & { }9.420 $\mu$s & 339 $\mu$s \\
    2 & 26 & { }6.620 $\mu$s & 344 $\mu$s \\
    3 & 14 & 18.564 $\mu$s & 520 $\mu$s \\
    4 &  40 & { }6.456 $\mu$s & 516 $\mu$s
    \end{tabular}
    \caption{Parameters used to implement $\pm\mathrm{x}$-gates.}
    \label{tbl:gate_parameters}
\end{table}



\subsection{Carbon Ramsey experiment }
By performing a Ramsey experiment we can determine the precession of the carbon-spin and its dephasing-time $T_2^*$.
By determining the precession frequencies it is possible to track phase evolution and use that to implement operations with arbitrary phase.
By measuring the precession frequency it is also possible to disprove our estimation for the hyperfine parameters.
We require the dephasing time in order to determine if enough operations can be applied to implement quantum algorithms \footnote{Change this sentence}.

In an ordinary Ramsey experiment a qubit is brought to the equator of the Bloch-sphere where it precesses for a time $\tau $ before it is read out along the x-direction.
A carbon-Ramsey experiment is similar but slightly more complicated as the nuclear spin cannot be controlled and read-out directly.
An uninitialized and an initialized version are depicted in \cref{fig:gate_circuit_nuclear_ramsey}.

In the initialized version of the carbon-Ramsey experiment the system is first initialized in the $\ket{0}_e\ket{X}_\mathrm{C}$-state.
The carbon is let to precess for a time $\tau$ before the $\ket{X}_\mathrm{C}$-state is read out.
During the free evolution the carbon rotates with a frequency of $\omega_L$ because the electron is in $\ket{0}_e$ and there is no coupling.

\begin{figure}[htbp]
        \centering
        \mbox{
        \Qcircuit @C=1em @R=.7em {
        \lstick{\ket{0}}          & \gate{\mathrm{y}}  & \ctrl{1}      & \qw & \multigate{1}{\tau}       &  \qw &\ctrl{1}          & \gate{\mathrm{-y}}  &  \meter \\
        \lstick{\rho_\mathrm{m}}         & \qw              &  \gate{\pm \mathrm{x}}     & \qw& \ghost{\tau}        & \qw & \gate{\pm \mathrm{x}}      & \qw       &\qw&}}
    \caption{Gate circuit depicting a carbon Ramsey with}
    \label{fig:gate_circuit_nuclear_ramsey}
\end{figure}


\subsubsection{Determining the precession frequency}

A conceptually more interesting variety is the uninitialized carbon-Ramsey experiment depicted in \cref{fig:gate_circuit_nuclear_ramsey_no_init}.
In the uninitialized version the system is described by \cref{eq:density_after_Ren} before the free evolution starts.
Because the electron is in a superposition of $\ket{0}$ and $\ket{1}$ the carbon-spin while evolve with two frequencies; $\omega_L$ for $\ket{0}_e$ and $\tilde{\omega} = \bm{|\omega_L + A |} $ for $\ket{1}_e$.

Similar to how the last part of the initialized carbon-Ramsey circuit reads out along the X-direction the last part of the uninitialized carbon-Ramsey reads out along the X-direction for the $\ket{Y}_e$ and along the $-$X-direction for the $\ket{-Y}_e$.
The phase picked up while show up as an oscillation between the $\ket{0}$ and $\ket{1}$ in the readout.
If the carbon has picked up no phase the the electron will point towards $\ket{1}$ in the readout.
Because the uninitialized carbon-Ramsey evolves with two frequencies we expect the measured oscillation to be the sum of two cosines as described by \cref{eq:carbon_ramsey_expected}. Where $ \tilde\omega =   \sqrt{(\omega_L+A_\parallel) ^2 + A_\perp^2} $.
\begin{equation}
    \tfrac{1}{4} \cos(\omega_L \tau ) +\tfrac{1}{4}  \cos (\tilde{\omega} \tau ) + \tfrac{1}{2}
    \label{eq:carbon_ramsey_expected}
\end{equation}

\begin{figure}[htbp]
    \begin{subfigure}[t]{0.49\textwidth}\centering
        \caption{}
        \includegraphics{Img/CarbonRamsey_C1.pdf}
        \label{fig:CR_C1}
    \end{subfigure}
    \begin{subfigure}[t]{0.49\textwidth}\centering
        \caption{}
        \includegraphics{Img/CarbonRamsey_C4.pdf}
        \label{fig:CR_C4}
    \end{subfigure}
    \caption{The uninitialized carbon-Ramsey experiment shows an oscillation that is the sum of two cosines due to the phase picked up during free evolution.
    \Cref{fig:CR_C1} depicts carbon-1 and \cref{fig:CR_C4} carbon-4.
    The measured frequencies were, for carbon-1: $\omega_{L,C1} = 2\pi\cdot 325.81 \pm 0.25$ and  $\tilde \omega_{\mathrm{C1}}= 2\pi\cdot 364.41 \pm 0.23$, and for carbon-4: $\omega_{L,C4} =  2\pi\cdot 325.94 \pm 0.40$ and $\tilde \omega_{\mathrm{C4}} = 2\pi\cdot 371.52 \pm 0.39 $.}
    \label{fig:Uninitialized_carbon_ramsey}
\end{figure}

\Cref{fig:Uninitialized_carbon_ramsey} shows the results for an uninitialized carbon-Ramsey experiment.
The data was fitted to a sum of two cosines in order to determine the frequencies.
The Larmor frequencies are $\omega_{L,C1} = 2\pi\cdot 325.81 \pm 0.25$kHz  for carbon-1 and  $\omega_{L,C4} =  2\pi\cdot 325.94 \pm 0.40$kHz for carbon-4.
Both the measured Larmor frequencies agree with the magnetic field of 304G within two standard deviations.

The $\tilde{\omega}$ frequency can be used to disprove the estimations for the hyperfine parameters of \cref{tbl:HF_par}, however if the measured values agree with the hyperfine estimation we cannot conclude that the estimations are correct.

For $\tilde{\omega}$ the following frequencies were measured: $\tilde \omega_{\mathrm{C1}}= 2\pi\cdot 364.41 \pm 0.23$kHz for carbon-1
and $\tilde \omega_{\mathrm{C4}} = 2\pi\cdot 371.52 \pm 0.39 $kHz for carbon-4.
Based on the estimated hyperfine parameters we expect $\tilde\omega_{\mathrm{C1}} \approx 364.7\mathrm{kHz}$ for carbon-1 and $\tilde \omega_{\mathrm{C4}} \approx 371.4 \mathrm{kHz}$ for carbon-4.
Both these values are in good agreement with experiment, a good indication that our hyperfine estimation is accurate.


\subsubsection{Measuring $T_{2,\mathrm{C}}^* $}
% Lange carbon ramseys van Hans sil01 140506 #53 en 56 +T2* analyse.
% hoeveel pulses voor de lange tau? belangrijk voor T2*
In order to know how many operations we can perform on a qubit we must know how long the signal stays coherent under normal operation.
In the case of controlling weakly carbons this is while decoupling the electron.

In order to determine carbon dephasing while decoupling the electron an uninitialized carbon-Ramsey was performed where the electron is decoupled during the free evolution time.
Because the electron is constantly flipped the carbon will precess with an average frequency of $\omega_{\mathrm{DD}} = (\omega_L +\tilde{\omega} )/2$.
By undersampling with a frequency slightly detuned from the precession frequency ($\omega_{\mathrm{DD}}$) a decaying cosine can be observed where the 1/e time of the envelope is equal to $T_2^*$.

\begin{figure}[htbp]
    \begin{subfigure}[t]{0.49\textwidth}\centering
        \caption{}
        \includegraphics{Img/Carbon1_T2star.pdf}
        \label{fig:T2star_carbon1}
    \end{subfigure}
    \begin{subfigure}[t]{0.49\textwidth}\centering
        \caption{}
        \includegraphics{Img/Carbon4_T2star.pdf}
        \label{fig:T2star_carbon4}
    \end{subfigure}
    \caption{Carbon-Ramsey experiment to determine $T_2^*$ for nuclei while decoupling the electron.
    The decays are fitted with a generalized normal distribution to determine $T_2^*$ and the exponent $n$.
    \Cref{fig:T2star_carbon1}, for carbon-1, $T_{2,\mathrm{C1}}^* =9.85 \pm   0.39 \mathrm{ms}$ and $n= 1.83 \pm 0.19$.
    \Cref{fig:T2star_carbon4}, for carbon-4,  $T_{2,\mathrm{C4}}^* =6.68 \pm   0.22 \mathrm{ms}$ and $n= 2.31 \pm 0.31$. } %would like to add that not limited by electron coherence due to DD.
    \label{fig:T2star_carbon}
\end{figure}

The decay for both carbons follows a Gaussian profile within uncertainty.
The coherence times measured were $T_{2,\mathrm{C1}}^* =9.85 \pm   0.39 \mathrm{ms}$ for carbon-1 and $T_{2,\mathrm{C4}}^* =6.68 \pm   0.22 \mathrm{ms}$ for carbon-4.



