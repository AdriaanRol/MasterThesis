\section{Characterizing weakly-coupled carbon spins}

This section will explain how a weakly-coupled spin can be controlled using the conditional rotation of the carbon spin that occurs when on resonance (\cref{eq:res_dip_loc}).
It will start by explaining how to do basic gate operations in the ideal case of being perfectly on resonant and not interacting with any nuclear spins.
After that we will explain how a carbon-spin in a mixed state can be initialized

\subsection{Basic operations}
As was explained in \cref{sec:controllingacarbonthroughdynamicaldecoupling} nuclear spins perform a rotation along two anti-parallel axes when subjected to a dynamical decoupling sequence on a resonant condition given by \cref{eq:res_dip_loc}.
The angle of rotation can be controlled by choosing the number of pulses of the decoupling sequence.

By choosing the number of pulses such that all coherence is lost when performing a dynamical decoupling spectroscopy measurement on the resonance, a rotation of $\pi/2$ in the clockwise direction is performed when the electron is in the $\ket{0}$-state and a $\pi/2$-rotation in the counterclockwise direction when the electron is in the $\ket{1}$-state.
We define the axis of rotation of this operation as the $x$-axis.

We call this conditional rotation the $\pm \mathrm{x}$-gate and it forms the basis of our control over weakly coupled spins. \Cref{fig:gate_circuit_pm-x} shows how we depict the $\pm \mathrm{x}$-gate in a circuit-diagram.
By letting the phase of the carbon evolve we are able to apply operations on the carbon-spin with arbitrary phase.
An unconditional gate can be implemented by placing the electron in an eigenstate before performing the $\pm\mathrm{x}$-operation.

\begin{figure}[htbp]
    \centering
        \mbox{
        \Qcircuit @C=1em @R=.7em {
         \lstick{\ket{\Psi}_e} &\ctrl{1}  &\qw\\
          \lstick{\ket{\Psi}_\mathrm{C}} &\gate{\pm \mathrm{x} }  &\qw}}
    \caption{The conditional x-gate ($\pm\mathrm{x}$). Performs an x-rotation on the carbon state ($\ket{\Psi }_\mathrm{C}$) when the electron is in the $\ket{0}_e$-state. It performs a $-\mathrm{x}$ rotation when the electron is in the $\ket{1}_e$-state.}
    \label{fig:gate_circuit_pm-x}
\end{figure}

Basic gates can be calibrated by sweeping the number of pulses $N$ when on resonance $\tau$.
In this manner carbon-1 and carbon-4 were found to perform the best $\pm\mathrm{x}$-gates.
The parameters used to implement $\pm\mathrm{x}$-gates are listed in \cref{tbl:gate_parameters}.

\begin{table}[htbp]
    \centering
    \begin{tabular}{cccc}
    Carbon &  $ N $ &  $\tau$ & total gate time\\ \hline
    1 &  18 & { }9.420 $\mu$s & 339 $\mu$s \\
    2 & 26 & { }6.620 $\mu$s & 344 $\mu$s \\
    3 & 14 & 18.564 $\mu$s & 520 $\mu$s \\
    4 &  40 & { }6.456 $\mu$s & 516 $\mu$s
    \end{tabular}
    \caption{Parameters used to implement $\pm\mathrm{x}$-gates.}
    \label{tbl:gate_parameters}
\end{table}



\subsection{Carbon Ramsey experiment }
By performing a Ramsey experiment we can determine the precession of the carbon-spin and its dephasing-time $T_2^*$.
By determining the precession frequencies it is possible to track phase evolution and use that to implement operations with arbitrary phase.
By measuring the precession frequency it is also possible to disprove our estimation for the hyperfine parameters.
We require the dephasing time in order to determine if enough operations can be applied to implement quantum algorithms \footnote{Change this sentence}.

In an ordinary Ramsey experiment a qubit is brought to the equator of the Bloch-sphere where it precesses for a time $\tau $ before it is read out along the x-direction.
A carbon-Ramsey experiment is similar but slightly more complicated as the nuclear spin cannot be controlled and read-out directly.
An uninitialized and an initialized version are depicted in \cref{fig:gate_circuit_nuclear_ramsey}.

In the initialized version of the carbon-Ramsey experiment the system is first initialized in the $\ket{0}_e\ket{X}_\mathrm{C}$-state.
The carbon is let to precess for a time $\tau$ before the $\ket{X}_\mathrm{C}$-state is read out.
During the free evolution the carbon rotates with a frequency of $\omega_L$ because the electron is in $\ket{0}_e$ and there is no coupling.

\begin{figure}[htbp]
        \centering
        \mbox{
        \Qcircuit @C=1em @R=.7em {
        \lstick{\ket{0}}          & \gate{\mathrm{y}}  & \ctrl{1}      & \qw & \multigate{1}{\tau}       &  \qw &\ctrl{1}          & \gate{\mathrm{-y}}  &  \meter \\
        \lstick{\rho_\mathrm{m}}         & \qw              &  \gate{\pm \mathrm{x}}     & \qw& \ghost{\tau}        & \qw & \gate{\pm \mathrm{x}}      & \qw       &\qw&}}
    \caption{Gate circuit depicting a carbon Ramsey with}
    \label{fig:gate_circuit_nuclear_ramsey}
\end{figure}


\subsubsection{Determining the precession frequency}

A conceptually more interesting variety is the uninitialized carbon-Ramsey experiment depicted in \cref{fig:gate_circuit_nuclear_ramsey_no_init}.
In the uninitialized version the system is described by \cref{eq:density_after_Ren} before the free evolution starts.
Because the electron is in a superposition of $\ket{0}$ and $\ket{1}$ the carbon-spin while evolve with two frequencies; $\omega_L$ for $\ket{0}_e$ and $\tilde{\omega} = \bm{|\omega_L + A |} $ for $\ket{1}_e$.

Similar to how the last part of the initialized carbon-Ramsey circuit reads out along the X-direction the last part of the uninitialized carbon-Ramsey reads out along the X-direction for the $\ket{Y}_e$ and along the $-$X-direction for the $\ket{-Y}_e$.
The phase picked up while show up as an oscillation between the $\ket{0}$ and $\ket{1}$ in the readout.
If the carbon has picked up no phase the the electron will point towards $\ket{1}$ in the readout.
Because the uninitialized carbon-Ramsey evolves with two frequencies we expect the measured oscillation to be the sum of two cosines as described by \cref{eq:carbon_ramsey_expected}. Where $ \tilde\omega =   \sqrt{(\omega_L+A_\parallel) ^2 + A_\perp^2} $.
\begin{equation}
    \tfrac{1}{4} \cos(\omega_L \tau ) +\tfrac{1}{4}  \cos (\tilde{\omega} \tau ) + \tfrac{1}{2}
    \label{eq:carbon_ramsey_expected}
\end{equation}

\begin{figure}[htbp]
    \begin{subfigure}[t]{0.49\textwidth}\centering
        \caption{}
        \includegraphics{Img/CarbonRamsey_C1.pdf}
        \label{fig:CR_C1}
    \end{subfigure}
    \begin{subfigure}[t]{0.49\textwidth}\centering
        \caption{}
        \includegraphics{Img/CarbonRamsey_C4.pdf}
        \label{fig:CR_C4}
    \end{subfigure}
    \caption{The uninitialized carbon-Ramsey experiment shows an oscillation that is the sum of two cosines due to the phase picked up during free evolution.
    \Cref{fig:CR_C1} depicts carbon-1 and \cref{fig:CR_C4} carbon-4.
    The measured frequencies were, for carbon-1: $\omega_{L,C1} = 2\pi\cdot 325.81 \pm 0.25$ and  $\tilde \omega_{\mathrm{C1}}= 2\pi\cdot 364.41 \pm 0.23$, and for carbon-4: $\omega_{L,C4} =  2\pi\cdot 325.94 \pm 0.40$ and $\tilde \omega_{\mathrm{C4}} = 2\pi\cdot 371.52 \pm 0.39 $.}
    \label{fig:Uninitialized_carbon_ramsey}
\end{figure}

\Cref{fig:Uninitialized_carbon_ramsey} shows the results for an uninitialized carbon-Ramsey experiment.
The data was fitted to a sum of two cosines in order to determine the frequencies.
The Larmor frequencies are $\omega_{L,C1} = 2\pi\cdot 325.81 \pm 0.25$kHz  for carbon-1 and  $\omega_{L,C4} =  2\pi\cdot 325.94 \pm 0.40$kHz for carbon-4.
Both the measured Larmor frequencies agree with the magnetic field of 304G within two standard deviations.

The $\tilde{\omega}$ frequency can be used to disprove the estimations for the hyperfine parameters of \cref{tbl:HF_par}, however if the measured values agree with the hyperfine estimation we cannot conclude that the estimations are correct.

For $\tilde{\omega}$ the following frequencies were measured: $\tilde \omega_{\mathrm{C1}}= 2\pi\cdot 364.41 \pm 0.23$kHz for carbon-1
and $\tilde \omega_{\mathrm{C4}} = 2\pi\cdot 371.52 \pm 0.39 $kHz for carbon-4.
Based on the estimated hyperfine parameters we expect $\tilde\omega_{\mathrm{C1}} \approx 364.7\mathrm{kHz}$ for carbon-1 and $\tilde \omega_{\mathrm{C4}} \approx 371.4 \mathrm{kHz}$ for carbon-4.
Both these values are in good agreement with experiment, a good indication that our hyperfine estimation is accurate.


\subsubsection{Measuring $T_{2,\mathrm{C}}^* $}
% Lange carbon ramseys van Hans sil01 140506 #53 en 56 +T2* analyse.
% hoeveel pulses voor de lange tau? belangrijk voor T2*
In order to know how many operations we can perform on a qubit we must know how long the signal stays coherent under normal operation.
In the case of controlling weakly carbons this is while decoupling the electron.

In order to determine carbon dephasing while decoupling the electron an uninitialized carbon-Ramsey was performed where the electron is decoupled during the free evolution time.
Because the electron is constantly flipped the carbon will precess with an average frequency of $\omega_{\mathrm{DD}} = (\omega_L +\tilde{\omega} )/2$.
By undersampling with a frequency slightly detuned from the precession frequency ($\omega_{\mathrm{DD}}$) a decaying cosine can be observed where the 1/e time of the envelope is equal to $T_2^*$.

\begin{figure}[htbp]
    \begin{subfigure}[t]{0.49\textwidth}\centering
        \caption{}
        \includegraphics{Img/Carbon1_T2star.pdf}
        \label{fig:T2star_carbon1}
    \end{subfigure}
    \begin{subfigure}[t]{0.49\textwidth}\centering
        \caption{}
        \includegraphics{Img/Carbon4_T2star.pdf}
        \label{fig:T2star_carbon4}
    \end{subfigure}
    \caption{Carbon-Ramsey experiment to determine $T_2^*$ for nuclei while decoupling the electron.
    The decays are fitted with a generalized normal distribution to determine $T_2^*$ and the exponent $n$.
    \Cref{fig:T2star_carbon1}, for carbon-1, $T_{2,\mathrm{C1}}^* =9.85 \pm   0.39 \mathrm{ms}$ and $n= 1.83 \pm 0.19$.
    \Cref{fig:T2star_carbon4}, for carbon-4,  $T_{2,\mathrm{C4}}^* =6.68 \pm   0.22 \mathrm{ms}$ and $n= 2.31 \pm 0.31$. } %would like to add that not limited by electron coherence due to DD.
    \label{fig:T2star_carbon}
\end{figure}

The decay for both carbons follows a Gaussian profile within uncertainty.
The coherence times measured were $T_{2,\mathrm{C1}}^* =9.85 \pm   0.39 \mathrm{ms}$ for carbon-1 and $T_{2,\mathrm{C4}}^* =6.68 \pm   0.22 \mathrm{ms}$ for carbon-4.


