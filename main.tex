\documentclass[a4paper,10pt,twoside,openright]{report}
\addtolength{\oddsidemargin}{-.5in}
\addtolength{\evensidemargin}{-.5in}
\addtolength{\textwidth}{3cm}
\addtolength{\topmargin}{-.875in}
\addtolength{\textheight}{1.75in}
% \textwidth 16.59cm
% \textheight 21.94cm

\usepackage{amsmath}
\usepackage{lscape}
\usepackage{graphicx}
\usepackage[margin=0pt,font=small,labelfont=bf,labelsep=endash]{caption}
\usepackage{pdfpages}
\usepackage{amssymb}
\usepackage[position=t,singlelinecheck=off]{subcaption}
\DeclareCaptionLabelFormat{bold}{\textbf{(#2)}}
\captionsetup{subrefformat=bold}
\usepackage{amsthm}
\usepackage{rotating}
\include{qcircuit} % Package used for typesetting quantum circuits. Included in directory
\usepackage{sectsty}
\usepackage[numbers,sort&compress]{natbib} % Natbib package, supports more fancy ways of citing
\usepackage{bm} %Allows for boldface of Greek characters in math mode.
\usepackage{placeins}
% \usepackage{layouts} %Used to print page-widths
\usepackage{tikz} %For drawing figures. Might make compilation slow.
\sectionfont{\sffamily\Large}
\subsectionfont{\sffamily}
\usepackage[Bjornstrup]{fncychap}
\ChTitleVar{\raggedleft\LARGE\sffamily\bfseries}
\ChNumVar{\fontsize{76}{80}\usefont{OT1}{phv}{m}{n}\selectfont}
%%%\setlength{\parindent}{0in}
\newcommand{\comment}[1]{} %Creates a command that does nothing in order to allow in-line comments
\setlength{\marginparwidth}{0pt}
\hfuzz=10pt %disables warnings for overfull hbox upto 10pt. Used for chapter headings
% \pdfsuppresswarningpagegroup=1 % Disables warnings for multiple pdfs with grouping on same page
\usepackage{fancyhdr}
\usepackage{hyperref} %Adds hyperlinks to references
\hypersetup{
    % bookmarks=false,         % show bookmarks bar?
    unicode=false,          % non-Latin characters in Acrobat’s bookmarks
    pdftoolbar=true,        % show Acrobat’s toolbar?
    pdfmenubar=true,        % show Acrobat’s menu?
    pdffitwindow=false,     % window fit to page when opened
    pdfstartview={FitH},    % fits the width of the page to the window
    pdftitle={Entanglement of Weakly-coupled Carbon-spins in Diamond},    % title
    pdfauthor={M.A. Rol},     % author
    pdfsubject={Quantum Computation},   % subject of the document
    pdfcreator={Adobe},   % creator of the document
    pdfproducer={M.A.Rol}, % producer of the document
    pdfkeywords={Entanglement} {Quantum computing} {Interference}{Parity Measurements}{Quantum Error Correction}, % list of keywords
    pdfnewwindow=true,      % links in new window
    colorlinks=true,       % false: boxed links; true: colored links
    linkcolor=darkgray,          % color of internal links
    citecolor=darkgray,        % color of links to bibliography
    filecolor=magenta,      % color of file links
    urlcolor=darkgray           % color of external links
}
\usepackage{enumerate}
\usepackage[capitalise]{cleveref} %Adds the cref command for clever referencing
% \usepackage{svg}

\begin{document}

\newenvironment{changemargin}[2]{%
\begin{list}{}{%
\setlength{\topsep}{0pt}%
\setlength{\leftmargin}{#1}%
\setlength{\rightmargin}{#2}%
\setlength{\listparindent}{\parindent}%
\setlength{\itemindent}{\parindent}%
\setlength{\parsep}{\parskip}%
}%
\item[]}{\end{list}}

% \includepdf{./Cover/cover_file.pdf}
% \newpage
% \thispagestyle{empty}
% \mbox{}
% \newpage

\begin{abstract}

Quantum error correction (QEC) is an essential ingredient for scaling up a quantum computer.
The nitrogen-vacancy (NV) center in diamond is a promising candidate for such a scalable quantum computer.
This thesis will first explain the tools available to control the NV-center and strongly coupled spins in its environment.
We will then demonstrate how the spin register can be extended by addressing weakly coupled carbon-spins.
The main result of this thesis is the creation of entanglement between two weakly coupled spins, for which initialization, control and read-out is required.
We demonstrate entanglement with a fidelity of  $F = 0.76 \pm  0.02$, well above the quantum limit for an entangled state.
The last chapter will provide an outlook on how the capabilities developed can be extended to implement QEC.

\end{abstract}

\tableofcontents

\chapter{Introduction}


%weakly coupled carbons in NV- environment provide a naturally occuring spin register making it a promising candidate for QC
% For large scale larger and  deterministically available spin registers are desirable
% here we show deterministic generation of entanglement paving the way for measurement based QEC.
%  Simultations indicate that it is possible to control even more Carbon atoms extending the register and performing full QEC

% An essential feature for a scalable quantum computer is the ability to correct for quantum errors.
% The Nitrogen-Vacancy center in diamond is a promising candidate for a scalable quantum computer.
% In order to perform

% This chapter will explain why building a quantum computer is cool/important/essential/someotherbigword
% Explain why quantum error correction is essential in achieving this goal
% Requirements to perform quantum error correction
% Explain NV natural candidate
% Requirements 3-5 qubits + entanglement.
% This thesis will demonstrate how to address weakly coupled carbon spins and create entanglement between them.

% Paragraph: Quantum is cool because exponential scale up allowing previously unattainable problems

%The Nitrogen-Vacancy (NV) center in diamond is a promising candidate for a scalable quantum computer.
% An essential feature for building a scalable quantum computer is the ability to correct quantum errors.
% A basic form of Quantum Error Correction (QEC)  (cite references, either in NV or global)
% A logical next step is to perform QEC based on repeated parity measurements. (needs clarification).
% In order to perform measurement based QEC one needs to be able to initialize, control, read-out and entangle up-to 5 qubits.

\section{Quantum Computing}
The idea of using a quantum mechanical system to simulate physics was first explored by Feynman\cite{Feynman1982Simulating}.
Because the number of operations required to simulate a quantum system scales exponentially with the number of particles in the system it is not feasible to classically simulate such a system.
By manipulating a quantum mechanical system directly this scaling problem can be circumvented.

It was the idea of quantum simulation that led to the idea of exploiting quantum effects to perform computations.
Shor's algorithm \citep{Shor1994Algorithms} is a prime example of how the advantageous scaling of quantum systems can be exploited toe perform more efficient computations.

Shor's algorithm is an algorithm for prime factorization, where the best known classical algorithms for prime factorization scale exponentially Shor's algorithm scales polynomially.
Because of  the speedup provided by Shor's and other quantum algorithms it is possible to solve classes of problems that were previously unsolvable, such as the breaking of RSA encryption that relies on prime factorization being a computationally hard problem.
%Look up TJ Ladd et al. 2010 nature. On exponential speedup and entanglement as key resource. Can I link entanglement as resource to the propagation of errors?
% Proof of concept exists (+ examples of basic algorithms) but key challenge is scaling it up
Shor's algorithm has been shown to work on a small scale quantum computer \cite{Vandersypen2001Experimental}.
However to take full advantage of the efficient scaling behaviour of quantum computations a scalable quantum computer is required.

\section{Quantum Error Correction}
One of the main challenges in creating a scalable quantum computer is the effect of errors.
Because a quantum computer uses entanglement between multiple qubits as a resource an error an a single qubit can quickly propagate trough the system.
Using quantum error correction (QEC) it is possible to correct for such errors.
The threshold theorem states the final error probability of a quantum algorithm on a large register can be made arbitrarily small, if the probability of an error on a single qubit is below a certain threshold and QEC is applied \citep{Mermin1990Extreme}.
This makes quantum error correction a stepping stone on the way to realizing a scalable quantum computer \citep{Nielsen2010Quantum}.

Experiments demonstrating a form of QEC have been demonstrated in a range of systems.
Codes correcting for one type of quantum error have been implemented with nuclear magentic resonance \citep{Cory1998Experimental,Moussa2011Demonstration}, trapped ions \citep{Schindler2011Experimental}  and superconducting qubits \citep{Reed2012Realization}.
Recently two groups have implemented three qubit gate based QEC using the nitrogen-vacancy (NV) center in diamond \citep{Taminiau2014Universal,Waldherr2014Quantum}.

Although schemes exist to implement a scalable quantum computer using a purely gate based architecture most approaches rely on correcting for quantum errors by performing repeated parity measurements. An example of such an approach is surface coding \citep{Fowler2012Surface}.
For this reason the present work will focus on performing the parity measurements required for such error correcting schemes.

QEC is similar to classical error correction trough majority voting.
In classical majority voting a single bit is encoded onto a \emph{logical} bit consisting of multiple regular bits.
When an error on a bit has occurred the original bit can be recovered by measuring all the individual bits and determining the original state by majority voting.

In QEC the quantum bit (qubit) is encoded onto a logical qubit consisting of multiple regular qubits.
A consequence of classical majority voting is that it destroys a quantum state because it requires all bits to be measured.
By measuring the parity of two qubits the difference between two states can be determined without measuring the individual states.
If two states are parallel in the parity basis the parity measurement will return a positive result, if they are anti-parallel a negative result.
Performing a parity measurement on multiple qubits allows quantum errors, such as a bit-flip, a phase-flip or the combination of both, to be diagnosed.

\Cref{fig:gate_circuit_3_qubit_QEC} shows a circuit diagram for three qubit measurement based QEC.
The state $\ket{\psi}_\mathrm{C2}$ is encoded onto a logical qubit consisting of qubits C1, C2 and C3.
A parity measurement is performed on qubits C1 and C2, and on C2 and C3,  by reading out an ancilla qubit $e$.
From the result of the parity measurements it can be determined on which qubit an error occurred.

\begin{figure}[htbp]
    \centering
    \mbox{
    \Qcircuit @C=1em @R=.7em {
        &&&& \control &\cw & \cw &\cw & \cw &\cw & \cw &\cw & \control \cw &\\
        \lstick{\ket{0}_e }        & \qw & \targ &\targ & \meter \cwx &  && \lstick{\ket{0}_e}     &\qw & \targ & \targ & \meter &  \cw \cwx  &\lstick{\ket{0}_e} &\qw&\qw &\qw &\rstick{\ket{0}_e} \\
        \lstick{\ket{0}_{\mathrm{C1}}}     &  \targ \qwx[1]& \ctrl{-1} &\qw & \qw &\qw&\qw& \qw &\qw & \qw &\qw & \qw & \multigate{2}{\mathrm{Corr.}} \cwx &\qw &\qw &  \targ \qwx[1] &\qw &\rstick{\ket{0}_{\mathrm{C1}}}\\
        \lstick{\ket{\psi}_{\mathrm{C2}}} &  \ctrl{1} & \qw &\ctrl{-2} & \qw &\qw&\qw&\qw  & \qw&\ctrl{-2} &\qw &\qw &\ghost{\mathrm{Corr.}}&\qw & \qw  &  \ctrl{1}&\qw &\rstick{\ket{\psi}_{\mathrm{C2}}}\\
        \lstick{\ket{0}_{\mathrm{C3}}}     & \targ   &\qw &\qw& \qw &\qw&\qw& \qw &\qw & \qw &\ctrl{-3} & \qw &\ghost{\mathrm{Corr.}}& \qw &\qw &  \targ&\qw &\rstick{\ket{0}_{\mathrm{C3}}}
        \gategroup{1}{3}{5}{13}{.7em}{--}  \
        % \gategroup{1}{3}{5}{12}{.7em}{_\}}
        % \gategroup{1}{13}{5}{13}{.7em}{_\}}
        % \gategroup{1}{1}{5}{2}{.7em}{_\}}
        % \gategroup{1}{14}{5}{14}{.7em}{_\}}
        }
    }
    \caption{Gate circuit for three qubit error correction. First the state of $C2$ is encoded onto three qubits.
    To diagnose an error an ancilla qubit is used to perform two parity measurements on the encoded qubits.
    The result of these measurements is used to determine what gates to apply to correct for the error if it occurred.
    These error correcting operations (in the dahsed lines) can be repeated to continuously protect the encoded qubits against quantum errors.
    The three qubit error correction code can correct for one type of quantum error. }
    \label{fig:gate_circuit_3_qubit_QEC}
\end{figure}

In order to implement measurement based error correction as depicted in \cref{fig:gate_circuit_3_qubit_QEC} we require control over three qubits and an ancilla that can be read out.
Furthermore we need to be able to perform parity measurements on these qubits and deterministically perform operations based on the outcome of these parity measurements to correct for errors.

\section{The Nitrogen Vacancy Center in Diamond}

Optical links \citep{Bernien2013Heralded,Pfaff2014Unconditional}
Nodes \citep{Taminiau2014Universal,Waldherr2014Quantum} (error correction)

Prominent approach to Scalable networks rely on repeated parity measurements (i.e. approach surface code. )

% Normally a source of errors (decoherence) weakly coupled carbons in the environment of the NV provide a naturally occuring register of qubits that can be used for QEC.

% Has been shown that they can be addressed and controlled (paper of Tim)

% Gate based QEC has been shown.
% Q: Why is this not full QEC? Don't need deterministic gates/FFWD?

% In this thesis we will show coherent control of weakly coupled carbons and deterministic entanglement creation.


% Notes, think of use of word parity and entanglement

%TODO_MAR: Revise part below
The Nitrogen Vacancy centre in diamond is a well investigated system\citep{Doherty2013NitrogenVacancy} and a promising candidate for quantum computation\citep{Childress2013Diamond}. In order to implement three qubit measurement based QEC we need three qubits plus ancillae that we can initialise, measure and conditionally perform operations on. These extra qubits are found in Carbon--13 atoms, which are normally a source of decoherence. These atoms can be addressed using a resonant decoupling sequence\citep{Taminiau2012Detection}.


\section{Goal of the project}
The Goal of the project is to create entanglement between weakly coupled carbon spins.
This is important because entanglement creation is a key component for QEC in particular and QI in general.

For this we need a b and c

% Gives motivation for doing parity measurements based on the importance of QEC
% Clearly ending in several key requirements for the experiments:
        %  Control of enough qubits /carbons
        %  Feed forward -> Deterministic gates and dynamical operations
        %  Parity measurements
\chapter{Electronic spins in Diamond}
\label{controlingspinsindiamond}
%TODO_MAR:  Status: Up for revision
% TODO_MAR: Fix definition of weakly coupled carbons


% TODO_MAR: Rewrite introduction of diamond spin control chapter

%Explain here how general experiment looks

It has been shown that the nuclear- and electron- spin-state of the NV- center can be initialized, controlled and read-out using microwave- and laser- pulses\citep{Robledo2011HighFidelity}. In these experiments two lasers that are resonant with transitions in the NV- center are used to initialize the electronic spin state.
One of these two lasers is used to read out the electronic spin state and an off-resonant laser is used to reset the system. Microwaves are used to drive transitions between the different nuclear and electronic spin states.
%Add figure explaining read out en init.

Strongly coupled nuclear spins can be initialized by conditionally rotating the electronic state to a state that is read out only if the Carbon is in the desired state, when the electronic state readout has a positive result the system is projected into the desired state. We call this Measurement Based Initialization (MBI).
%
%
%%%%%%%%%%%%%%%%%%%
%TODO_MAR: add picture of Sil, schematic (with laser only hits one line)
%%%%%%%%%%%%%%%%%%%
%
%
% Explain pump laser and excitation laser,
% Explain experiment starts with checking if Lasers are on resonance
% Explain Nitrogen initialization to remove term from the Hamiltonian
%
% Explain structure of typical experiment and experimental setup
% Setup, same as in old papers. State temperature 4K, state that magnetic field is used.
%Structure.
%CR-check , Init electron with lasers, Init Nitrogen, do fancy pulse sequence, readout, possibly feed forward, more pulses, RO again.
%
Our experiments are build around the same basic tools\comment{As the Robledo experiment? }.
Each experiment starts with a Charge-Resonance check \comment{Explain how CR-Check works? Ref?}that verifies if the lasers are still on resonance.
After that the Nitrogen spin state is initialized using MBI.\comment{In what state is the Nitrogen initialized?}
Once the system is initialized the actual experiment is performed.
An experiment consists of one or multiple blocks of microwave pulses and optical readouts.

All experiments were performed on a custom-build cryostat setup operating at liquid helium temperatures described in detail in \citet[chap.~3]{Bernien2014Control}. The setup was additionally outfitted with a movable neodymium magnet that applied a magnetic field of ~300G to the sample.

\section{Spin Control}
\label{spincontrol}

The electronic ground state Hamiltonian can be written as\citep{Pfaff2013Quantum}: \comment{Do I cite wolfgang or Hannes or both?}
 \begin{equation}
H_{GS} = \Delta {S_z}^2 + \gamma_e \mathbf{B} \cdot \mathbf{S}
\end{equation}

With zero field splitting $\Delta \approx 2.88 \mathrm{GHz}$  and gyro-magnetic ratio $\gamma_e  = 2.802$ MHz/G . In this expression the interactions with the nitrogen nucleus and the carbon spin bath are not included. By applying a magnetic field $B_z$ a long the NV axis the degeneracy of the  $m_s =\pm1$ states is lifted by the Zeeman effect. We define our electronic qubit  by the two level system with  $m_s=0:=|0\rangle$ and $m_s = +1 := |1\rangle$.

On the Bloch-sphere the state vector rotates around the quantization axis with a frequency depending on the energy splitting between the two states; the Larmor frequency.
For the NV-electronic spin the Larmor frequency is given by:  $\omega_L =\Delta - \gamma_e {B_z} $.

By applying an external field a term is effectively added to the Hamiltonian, changing the quantization axis and thereby its evolution. By applying microwaves with the right frequency this can be used to selectively drive the transition from the  $|0\rangle$ state to the $|1\rangle$state\citep{Jelezko2004Observation}.

% \section{Dynamical decoupling}
% DONT need to explain general dynamical decoupling?
%Explain what dynamical decoupling is, decoupling from environment by 'inverting' the enviroment thus reducing influence of decoherence causing carbons.
%Show figure from Tim's paper showing extended electron coherence times.
% Make bridge to next chapter by refering paper again showing that it's possible to control carbons in this way by resonantly doing this

% Chapter elaborates on general electronic spin control and the architecture of a typical experiment
%TODO_MAR:  Status: Up for revision

\chapter{Weakly-coupled Carbon Spins}
Similar to how the electronic spin state can be controlled by adding and removing a term to the Hamiltonian we can also control the state of a carbon-13 atom.
This chapter will start by providing theoretical background on the hyperfine coupling between carbon spins and the NV-center, before it will explain how nuclear spins can be addressed.
The next section will explain and show characterization of the nuclear spin environment.
The last section will explain how carbon-spins can be controlled and will demonstrate initialization control and readout of weakly coupled carbon spins.
\section{Hyperfine Coupling}
The Hamiltonian of the nuclear spin depends on the electronic spin state\citep{Taminiau2014Universal}.
A hyperfine term describing the interaction between the electronic spin and the nucleus is not present when the electron is in the $m_s = 0$ state :
 \begin{eqnarray}
H_0= \gamma_C B_z I_z \\
H_1 = \gamma_C B_z I_z +H_{\mathrm{HF}}
\end{eqnarray}

The hyperfine ($H_{\mathrm{HF}}$) term consists of a contact term and a dipole term.
The contact term results from an overlap between the electronic- and carbon- wave-functions making it negligible for all but the carbon-spins closest to the NV- center.
Hyperfine strengths between the NV-center and carbons on close-by lattice sites have been measured\citep{Smeltzer201113} and calculated\citep{Gali2008Ab,Gali2009Identification}, but as we are only interested in weakly coupled carbons we can safely neglect the contact term of the hyperfine-interaction.

We define a carbon to be weakly coupled when its hyperfine coupling strength is much smaller than the Larmor frequency.
For all practical magnetic fields this does not include the carbons that have a contact term in their hyperfine-interaction.
For these carbons the hyperfine-term is equal to the dipole-term and is given by\citep{Lange2012Quantum}:

\begin{equation}
H_{\mathrm{dip}} = \frac{\mu_0 \gamma_e \gamma_C \hbar^2 }{4 \pi r^3} [ \bm{S \cdot I} - 3 (\bm S \cdot \hat{n_{\mathrm{hf}}})(\bm I \cdot \hat{n_{\mathrm{hf}}})]
\end{equation}

From this equation the parallel and orthogonal components of the Hyperfine interaction ($H_{\mathrm{dip}} = A_\parallel I_z + A_\perp I_x $), with respect to the NV-axis along the z direction, can be derived to be:
 \begin{eqnarray}
A_\parallel= - \frac{\mu_0 \gamma_e \gamma_C \hbar^2 }{4 \pi r^3} \left(3\cdot \frac{z^2}{r^2}-1\right)\\
 A_\perp =  -\frac{\mu_0 \gamma_e \gamma_C \hbar^2 }{4 \pi r^3}\left( 3\cdot\frac{\sqrt{x^2+y^2}\cdot z}{r^2}\right)
\end{eqnarray}

\section{Addressing Weakly-coupled Carbons through Dynamical Decoupling}
\label{controllingacarbonthroughdynamicaldecoupling}

\begin{figure}[htbp]
\centering
\includegraphics[keepaspectratio,width=0.2\textwidth,height=0.75\textheight]{./img/QuantizationAxis.png}
\caption{Flipping the electron spin from the  $m_s=0$ to the $m_s= +1$ state changes the quantization axis of $^{13}\mathrm{C}$ nuclear spins. For  $m_s=0$ spins precess about $\bm{\omega_L}$. For  $m_s=+1$ spins precess about a distinct axis $\bm{\tilde{\omega}}=\bm{\omega_L} +\bm{A}$.}
\label{fig:quantax}
\end{figure}

Spins precess about a quantization axis along $ \bm{\tilde{\omega}}$ with a frequency $|\bm{\tilde{\omega}}|$. We call $ \bm{\tilde{\omega}} $ the quantization-vector. When the electron is in the $m_s=0$ each nuclear spin precesses about $\bm{\tilde{\omega}} = \bm{\omega_L}$ with the Larmor frequency. The magnetic field is aligned along the quantization axis of the NV- center and defined as the z-direction.When the electron is in the $m_s=+1$ state nuclear spins precess about a distinct axis $\bm{\tilde{\omega}}=\bm{\omega_L} +\bm{A}$ \citep{Taminiau2012Detection}. The hyperfine interaction $\bm{A}$ depends on the position of that particular nuclear spin relative to the NV- center.

To understand how a carbon-13 atom can be controlled it is useful to consider three situations. In the first situation the $\bm{\omega_L}$ and $\bm{A}$ point in the same direction. In the second situation $\bm{\omega_L}$ and $\bm{A_\perp}$ are of comparable magnitude, resulting in a large angle between the quantization axes. In the last situation $|\bm{A}|$ is small compared to  $\bm{|\omega_L|}$ resulting in a small angle between the quantization axes.

When applying a decoupling sequence with N\slash 2 decoupling units of the form {$\tau - \pi -2\tau-\pi-\tau$}, where $\tau$ is a wait time between pulses and $\pi$ is a $\pi$-pulse that flips the electron-state, the nuclear spin alternately rotates around the  $\bm\omega_L$ and the $\bm{\tilde{\omega}}$ axis. The net result of one such decoupling sequence is a rotation around an axis $\bm{\hat{\mathrm{n_i}}}$ by an angle $\phi$. Where $\bm{\hat{\mathrm{n_i}}}$ depends on the initial state of the electron: $\bm{\hat{\mathrm{n_0}}}$ when the electron starts in $m_s = 0$ and $\bm{\hat{\mathrm{n_1}}}$ when the electron starts in $m_s = +1$~\citep{Taminiau2012Detection}.

When $\bm{\omega_L}$ and $\bm{A}$ point in the same direction, the net rotation axis is independent of the initial electron-state making it impossible to use the electron to control the carbon-13 atom using this decoupling sequence.

In the case where $\bm{\omega_L}$ and $\bm{A_\perp}$ are of comparable magnitude the net rotation axes $\bm{\hat{\mathrm{n_i}}}$ are strongly dependent on the initial electron-state for almost any $\tau$. Having one of these carbon atoms can make it hard to selectively control other carbons as there are very few inter-pulse-delays $2\tau$ for which only the carbon atom without the strong orthogonal-hyperfine is affected.

When considering the case where $\bm{|A|}$ is small compared to  $\bm{|\omega_L|}$ the net rotation axes  $\hat{n_0}$ and $\hat{n_1}$ are practically parallel and the nuclear spin undergoes an unconditional evolution. Only when the inter-pulse delay is precisely resonant with the spin dynamics the axes are anti-parallel leading to a conditional rotation\citep{Taminiau2012Detection}. The resonant condition occurs at:

 \begin{equation}
\tau = \frac{(2k+1)\pi}{2 \gamma_C B_z + A_\parallel}
\label{eq:res_dip_loc}
\end{equation}

And for $\omega_L \gg |\bm{A}|$ the dip has a width of:

 \begin{equation}
\Delta = \frac{A_\perp}{2\cdot (\gamma_C B_z)^2}
\label{eq:res_dip_width}
\end{equation}

If  $\hat{n_0}$ and $\hat{n_1}$ are not parallel, the resulting conditional rotation of the nuclear spin generally entangles the electron and nuclear spins. As a result, for an unpolarized nuclear spin state, the final electron spin state is a statistical mixture of $|x\rangle$ and $|-x\rangle$ when starting from the $|x\rangle$  state. Where the probability that the initial state is preserved is given by \cref{eq:contrast_to_probability}. The contrast $M_j$ for a single nuclear spin is given by \cref{eq:contrast_single_carbon_spin}\citep{Taminiau2012Detection}.

\begin{equation}
\label{eq:contrast_to_probability}
P_x = (M+1)/2
\end{equation}

\begin{equation}
\label{eq:contrast_single_carbon_spin}
M_j = 1-(1 - \hat{\bm{\mathrm{n_0}}} \cdot \hat{\bm{\mathrm{n_1}}}) \sin^2 \frac{N\phi}{2}
\end{equation}

%alpha = \tilde{\omega} \tau
%beta = (\omega_L \tau)
% mz = (\frac{ A_ \parallel + \omega_L }{ \tilde{ \omega}})
\begin{equation}
\label{eq:vec_term}
    1 - \hat{\bm{\mathrm{n_0}}} \cdot \hat{\bm{\mathrm{n_1}}} =  \frac{A_\perp ^2}{\tilde{\omega^2}} \frac{(1- \cos{(\tilde{\omega} \tau)})(1-\cos{(\omega_L \tau)})} {1 +\cos{(\tilde{\omega} \tau)}\cos{(\omega_L \tau)} - (\frac{ A_ \parallel + \omega_L }{ \tilde{ \omega}}) \sin{(\tilde{\omega} \tau)}\sin{(\omega_L \tau)}}
\end{equation}
\begin{equation}
\label{eq:angle_term}
    \phi =  \cos^{-1}\left(\cos(\tilde{\omega} \tau) \cos(\omega_L \tau)-\left(\frac{ A_ \parallel + \omega_L }{ \tilde{ \omega}}\right) \sin(\tilde{\omega} \tau)\sin(\omega_L \tau)\right)
\end{equation}

\section{Characterizing the Nuclear-spin environment}
% Should add some sort of introduction as to what is in this section
\subsection*{Dynamical Decoupling Spectroscopy}
In reality the electron is not interacting with a single carbon but with a bath of carbon atoms. When the electron interacts with multiple carbons at the same time the contrast $M$ is given by the product of all individual values $M_j$ for each individual spin $j$. In order to selectively control one carbon the electron should not entangle with any other carbon when addressing it.

\begin{equation}
\label{eq:prod_multiple_spins}
    M = \prod_{j}{M_j}
\end{equation}
%Explain how fingerprint works and how it characterizes how many carbon we can control.
% IF selective, M goes all the way down. If a lot go to 0.5 as all coherence is lost.

To identify promising resonances for carbon control a dynamical decoupling spectroscopy\citep{Taminiau2012Detection} or fingerprint is performed. In a fingerprint experiment the electron is prepared in the $|X\rangle = |0\rangle +|1\rangle$ state. It is subjected to a decoupling sequence consisting of N/2 blocks of the form {$\tau - \pi -2\tau-\pi-\tau$}, and concluded by measuring $\langle X\rangle $. The fingerprint is the result of many repetitions for a range of inter-pulse delays $2\tau$.

\subsection*{Contributions of different spins }% Needs a better name

A narrow dip in the fingerprint spectrum is an indication of a selectively controllable carbon.
By sweeping the number of $\pi$-pulses on such a dip it can be verified if it corresponds to a single carbon. If entanglement is created with a lot of spins at once all coherence is lost and contrast will go to 0. Only if no entanglement is created with any other carbon can the contrast be sweeped to -1. %This weird wording is used because it is possible that the response of another carbon is non existent exactly when the first one reaches -1. Maybe also add the in between case for few spins?

Because Carbon-13 atoms are randomly distributed in diamond there is a wide range of possible hyperfine strengths.
Most carbon-spins have very similar hyperfine-interaction strengths as they are relatively far away from the NV-center. This causes their resonances to overlap manifesting itself as a broad feature with little coherence in the fingerprint. We identify this response as the spin-bath.
% reality most far away -> similar strengths ->  spin bath response

Spins that have a stronger than average hyperfine-interaction show up outside or at the edge of the spin-bath response. Going to larger $\tau$ separates resonances further as their order $k$ increases, allowing for control of more spins.
 As computations are fundamentally limited by the coherence time there is a limit to the resonance-order that can be used to address carbons.
 Additionally some of the relatively strong-coupled spins also have a strong orthogonal component of the hyperfine interaction. This orthogonal-component causes a broad response, effectively blocking a large range of $\tau$ from being used to control other spins.

\subsection*{Effect of the magnetic field}
% need the stronger
% sometimes one that is to strong -> increase B-field
Both these issues can be alleviated by increasing the magnetic field.
By increasing the magnetic field the Larmor frequency can be made much larger than the orthogonal components of the hyperfine interactions causing the broad resonances to disappear.
% until the orthogonal hyperfine-components of all spins are small relative to the Larmor frequency the broad resonances that cause large parts of the spectrum to become useless disappear.
Additionally increasing the magnetic field causes resonances to move closer to $\tau =0$, while at the same time becoming narrower, allowing higher order resonances to be addressed within the coherence time. % use calculations of Tim to show that number of resonances that can be addressed increases linearly with magnetic field.

% to much B-field -> to narrow dips, cannot address anymore (tech limitation)
Increasing the magnetic field will not always improve the situation. When the magnetic field is too strong the resonances become narrower than the resolution of the Arbitrary Waveform Generator used to generate the pulses that address the resonances, making it impossible to address these resonances effectively.
Simulations that were performed \cref{chap:addressable_carbon_sims} indicate that for diamond with a natural concentration of carbon-13 there is a broad range between 400G and 1400G where the magnetic field is optimal.

There are also practical limitations to how much the magnetic field can be increased. In order to control carbon-spins we must still be able to coherently initialize, control and read-out the electronic spin state. Because the transitions used for read-out and initialization depend on strain and magnetic field\citep{Hensen2011MeasurementBased}, care must be taken when measuring at different magnetic fields that states do not mix in the excited state.
This combined with the fact that few experiments have been performed at high magnetic field and low temperature make it more practical to settle for a moderate magnetic field.
%Still misses point not practical to compensate for extremely strongly coupled spins.
% Optimum B-field range 300-700
%sims indicate optimum B? of behouden voor appendix?
% Additional problems with B. Stability? -> appendix. transitions in excited states?


\subsection*{Identifying Individual Carbon-spins}
To identify spins a dynamical decoupling spectroscopy was performed on the sample with N = 8, 16, 32 and 64 pulses. For N =8,16 and 32 this was done between $\tau = 2 \mu \mathrm{s}$  and $72 \mu \mathrm{s}$ and for N = 64 this was done up to $\tau = 52 \mu \mathrm{s}$.
We identify distinct features in the fingerprint and try to assign different hyperfine-couplings to them such that the computed data for these spins fits the measured data as well as possible.
The response of a single spin is computed using \cref{eq:contrast_single_carbon_spin}.
13 spins were identified using this method.

\Cref{fig:FP} shows a subset of the fingerprint data acquired for this thesis. \Cref{tbl:HF_par} shows the estimated hyperfine parameters of the 4 carbon spins with the strongest coupling.
All estimated hyperfine parameters and a link to the full fingerprint measurements can be found in \cref{chap:Fingerprint_data_appendix}.

The broad collapse due to the spin bath is clearly visible at $\tau/(4 \tau _L)$ for odd m.
The most prevalent feature of the spectrum is a strong oscillation between the spin-bath collapses.
This oscillation can be explained by a single carbon that has a strong orthogonal hyperfine coupling, labeled spin-2 in our analysis.
Due to the nature of spin-2 it is hard to find other carbons that can be coherently controlled.

Nonetheless there are still some distinct peaks at the edge of the spin-bath collapse. When going to higher orders we see these peaks separate from the spin-bath response.
We find that we can address the 4 spins listed in \cref{tbl:HF_par}. % Remove this sentence?


% Table with T2*, HF_parr, HF_orth. Values can be found in QEC LT/Simulations/Hans_Sil01_Spin_Control
% B_field = 304.22G
\begin{figure}[t]

    \begin{subfigure}[t]{\textwidth}\centering
        \centering
        \includegraphics{Img/fingerprint16.pdf}
        %TODO_MAR: use annotations to add labels in the graphs using matplotlib
        \caption{Fingerprint of Hans Sil01 at B = 304.12G for N=16 pulses.Larmor revival clearly visible. }
        \label{fig:FP16}
    \end{subfigure}

    \begin{subfigure}[t]{\textwidth}\centering
        \centering
        \includegraphics{Img/fingerprint32.pdf}
        %TODO_MAR: use annotations to add labels in the graphs using matplotlib
        \caption{Fingerprint of Hans Sil01 at B = 304.12G for N=32 pulses.Larmor revival clearly visible. }
        \label{fig:FP32}
    \end{subfigure}
    \caption{Fingerprints. \comment{NOTE TO SELF: remove fitting and add notes like in Julia poster. use tau/tau_l scale  }}
    \label{fig:FP}
\end{figure}




\begin{table}[htbp]
    \begin{tabular}{cllll}
    Carbon & \quad \quad  $A_{\parallel} $ & \quad \quad $A_{\perp}$ \\ \hline
    1         & $2 \pi \cdot$30.0 kHz             & $2 \pi \cdot$80.0 kHz                \\
    2         & $2 \pi \cdot$27.0 kHz             & $2 \pi \cdot$28.5 kHz              \\
    3         & $2 \pi \cdot$-51.0 kHz          & $2 \pi \cdot$105.0 kHz              \\
    4         & $2 \pi \cdot$45.1 kHz           & $2 \pi \cdot$20.0 kHz                \\
    \end{tabular}
    \caption{Estimated hyperfine parameters for spins 1 to 4 in \cref{fig:FP}.}
    \label{tbl:HF_par}
\end{table}

\subsection*{Measuring Precession Frequencies}
Lorem ipsum dolor sit amet, consectetur adipisicing elit, sed do eiusmod
tempor incididunt ut labore et dolore magna aliqua. Ut enim ad minim veniam,
quis nostrud exercitation ullamco laboris nisi ut aliquip ex ea commodo
consequat. Duis aute irure dolor in reprehenderit in voluptate velit esse
cillum dolore eu fugiat nulla pariatur. Excepteur sint occaecat cupidatat non
proident, sunt in culpa qui officia deserunt mollit anim id est laborum.
% Ramsey experiment to measure coupling strengths
% Needs relation between frequency and parralel component
% spins 1 and 4 best


\begin{figure}[htbp]
    \begin{subfigure}[t]{0.49\textwidth}\centering
    \includegraphics{Img/CarbonRamsey_C1.pdf}
    \caption{Nuclear Ramsey of Carbon 1} \label{fig:CR_C1}
    \end{subfigure}
    \begin{subfigure}[t]{0.49\textwidth}\centering
        \includegraphics{Img/CarbonRamsey_C4.pdf}
        \caption{Nuclear Ramsey of Carbon 4}
        \label{fig:CR_C4}
    \end{subfigure}
    \caption{Nuclear Ramsey experiment wit}
\end{figure}



\section{Controlling weakly coupled carbons trough the electronic spin}
% Section containing theory (Gate circuits) on how to initialize and readout carbons

Explain how carbon control works in theory.
Explain how a conditional and unconditional gate can be performed.
Explain initialization on gate level, refer to appendix for calculations.
Explain Readout.



\section{Carbon Initialization \& Readout}
% TODO_MAR: Discuss naming of sec: Carbon Init&RO and Carbon Tomo
%  Section containing experimental results (Tomographies)
%  Should emphasize difficulty in seperating initialization and RO fidelity, what is not working? Is it working?
Show results that demonstrate carbon control.




% Consists of several sub chapters
    %  theory of how weakly coupled carbons can be controlled
    %  Experiments used to characterize the environment and carbons
    %  Theory of initialization and readout of carbons
    %  Experiments showing single and multiple carbon initialization and readout
\chapter{Deterministic Parity Measurements}


\section{Entanglement}
%  Section gives definition of entanglement and refers to appendix for derivation of entanglement wittness.

\section{Verification of Entanglement}
% Section gives Results hopefully showing F>0.5 to Bell state with a cheer
%  Don't forget to elaborate where the error bars come from; statistical and does not include gate errors

%  Entanglement verification as test to check both feed forward and parity measurements
% Experiment (hopefully) showing F>0.5 to Bell state.
\chapter{Outlook: towards Quantum Error Correction}
This thesis has demonstrated the probabilistic creation of entangled states between weakly-coupled carbon spins.
This chapter will provide an outlook on how the tools developed can be used to implement quantum error correction.
It will also discuss the feasibility of using weakly-coupled carbon spins as a means to enlarge NV-spin registers in general

\section{Direct outlook, deterministic operations/what needs to be done/can be done}

\section{Simulations, how many carbon can we control? }



Moved from identification section:
\subsection{Effect of the magnetic field}

There are significant advantages to increasing the magnetic field when attempting to address weakly coupled carbons.
By increasing the magnetic field the Larmor frequency can be increased, reducing the number of carbons that are in the complex regime.
This causes the broad oscillating resonances to disappear allowing more carbons to be addressed.

Although increasing the magnetic field can improve the situation it is not always possible or desired.
When the magnetic field becomes too strong too strong the resonances become narrower than the resolution of the Arbitrary Waveform Generator used to generate the pulses that address the resonances, making it impossible to address these resonances effectively.
Simulations were performed (see \cref{chap:addressable_carbon_sims}) that indicate that for a natural carbon-13 concentration there is a range between 400G and 1400G where the magnetic field is optimal for controlling weakly coupled spins.

Besides the spin environment there are other factors affecting the choice for magnetic field.
Because the optical transitions used for readout and initialization depend on strain and magnetic field field\citep{Hensen2011MeasurementBased}, care must be taken when measuring that states do not mix in the excited state.
This combined with the fact that few experiments have been performed at high magnetic field and low temperature make it more practical to settle for a more moderate magnetic field of 300G.


%%%%%%%%%
Idem:

Spins that have a stronger than average hyperfine-interaction show up outside or at the edge of the spin-bath collapse.
Spins that are in the basic regime show up as a narrow dip.
Going to larger $\tau$ separates these dips further as the order of the resonance $k$ increases.
By looking at larger $\tau$ it is possible to resolve and address more resonances.
Several spins in the basic regime have been identified 3 of these are visible as colored lines in \cref{fig:FP}.
As computations are fundamentally limited by the coherence time there is a limit to the resonance-order that can be used to address carbons, making it impossible to resolve all weakly coupled spins.

Besides the carbons in the basic regime there are also weakly-coupled carbons that are more strongly coupled.
When a carbon in the complex regime is present in the NV-center this manifests itself as a resonance with strong oscillations on the side. Such a feature is also clearly visible in \cref{fig:FP}. We have identified the oscillations in the fingerprint as belong to a single spin which is denoted by the red line.

When a weakly coupled carbon in the complex regime is present a significant part of the fingerprint spectrum is inaccessible for controlling other carbons making them an undesired feature when attempting to control weakly coupled carbon spins.




%  Outlook what more is needed for QEC
%  Short term outlook that indicates usefull 3 qubit QEC circuits
% Simulations showing long term outlook for quantum nodes


%%% FINAL stuff
% \input{references.tex}

%%% appendices
\appendix
\chapter{Entanglement wittness}
% Derivation of F>0.5 to Bell state as entanglement wittness


%%% ACKNOWLEDGEMENTS
\chapter*{Acknowledgements}

% Tim for the immensely usefull feedback on my drafts, His interesting and for me new perspectives on what we are doing. Great coaching

% Julia.
% Ronald trust

\bibliographystyle{plain}
\bibliography{Remote}
\end{document}
