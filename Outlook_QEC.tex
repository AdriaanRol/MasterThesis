\chapter{Towards Quantum Error Correction}
This thesis has demonstrated the creation of entanglement between weakly coupled carbon spins.
This project was motivated by the use of



A crucial steps towards a scalable quantum computer.
next step implementing QEC.
This will

This thesis has demonstrated the probabilistic creation of entangled states between weakly-coupled carbon spins.
This chapter will provide an outlook on how the tools developed can be used to implement quantum error correction.
It will also discuss the feasibility of using weakly-coupled carbon spins as a means to enlarge NV-spin registers in general

\section{Direct outlook, deterministic operations/what needs to be done/can be done}
What is required to do detrministic gates?
Branching
FFFWD

Motivation for feed forward/conditional operations.
Currently conditionalizing on outcome, NOT post selecting.

Extra required, calculate what operations must be done to correct for opposite measurement outcome.
Integrate into experiment.


How to do deterministic?
By implementing feed forward it is possible to perform deterministic operations.
An entangled state can be deterministically created by


Advantages of deterministic
An additional advantage of performing deterministic operations is that it can significantly speed up the rate at which experiments can be performed.
There is a limit as all branches of an experiment


\section{Simulations}
In order to implement QEC control more than the two qubits that can be addressed now are required.
To correct for a single type of quantum error 3 qubits are required.
To implement any type of quantum error at least 5 qubits are required.
By having more qubits more elegant algorithms exist for correcting quantum errors.


To determine what the optimal conditions are with respect to the number of addressable carbons simulations were performed.
To determine if enough carbons can be controlled to implement QEC simulations were performed.

In the simulations it was determined how many weakly coupled carbons can be addressed using dynamical decoupling for a range of magnetic fields and several different concentrations.

This was done by collecting statistics on 1000 simulated NV-centers.
All these simulated NV-centers had an environment with randomly placed carbons.
For all these carbons the hyperfine interaction with the NV-center and its resonance conditions were calculated.
For these resonant conditions it was determined if a carbon could be controlled.

In the simulations a carbon is considered to be controllable if it matches certain conditions.
To be considered controllable it must be possible to implement two consecutive $\pm \mathrm{x}$ gates with 90\% fidelity, where the total gate time must be shorter than $T_{\mathrm{max}}$ using a resonance located at a time shorter than $\tau_{\mathrm{max}}$.

$T_max$ to simulate coherence time.  assumptions for coherence times (10 gates within)

Reject sample if big  coupled (implies higher statistics)
Max resonance time (to look for) (optional argument electron coherence).

Samples can be fabricated at different
Additional parameter that can be varied, carbon-13 concentration.
Longer coherence times for lower C
Weaker coupled carbons -> optimum at lower field.
Also non trivial.



In order to address weakly coupled carbons using dynamical decoupling their resonances must not overlap.
It is clear to see that it does not work at low field (?) because of the presence of multiple spins in the complex regime.
By increasing the magnetic field these can be moved to the simple regime.
As the field is increased the resonance become both narrower and move closer to the origin.
There are limits to how narrow a resonance can be and still be addressable.
If to close to origing (higher order) .
Not trivial, clear there is an optimum.


By increasing the magnetic field more carbons move from the complex regime (\cref{fig:coupling regimes}) to the basic regime.
At the same time all resonances become narrower ($\cref{eq:res_dip_width}$) while moving closer together in $\tau$ (\cref{eq:res_dip_loc}).

Closer -> possible to address higher orders? (because limited by coherence time)
Less complex regime -> more addressable

narrower -> AWG resolution
Tau shorter than pulsewidht

Caveat -> Large spread, impossible to predict about an individual carbon. Possible to say about expectation.

\begin{figure}[htbp]

    \begin{subfigure}[t]{0.49\textwidth}\centering
        \caption{}
        \begin{tikzpicture}
            \node[anchor=south west,inner sep=0] at (0,0) {\includegraphics{Img/Simulations_avgN_vs_Bfield.pdf}};
            \node[font=\small, text = blue] at (5.0,4.90)  {$\mu = 1.10\%$};
            \node[font=\small, text = green] at (5.0,4.6)  {$\mu =0.33\%$};
            \node[font=\small, text = red] at (5.0,4.30)  {$\mu =0.11\%$};
        \end{tikzpicture}
        \label{fig:Simulations_avg_n_vs_bfield}
    \end{subfigure}
    \begin{subfigure}[t]{0.49\textwidth}\centering
    \caption{}
    \begin{tikzpicture}
        \node[anchor=south west,inner sep=0] at (0,0) {\includegraphics{Img/Simulations_Histogram_vs_Bfield.pdf}};
        \node[font=\small, text = black] at (5.4,4.90)  {$B = 700\mathrm{G}$};
        \node[font=\small, text = black] at (5.4,4.60)  {$\mu = 1.1\%$};
    \end{tikzpicture}
    \label{fig:simulations_histogram_vs_Bfield}
    \end{subfigure}
    \caption{aapsdfklh }
    \label{fig:FP}
\end{figure}

Simulations only look at samples withiout large (>200kHz) HF couplings  carbon spins.

In the simulations 3 or more carbons can be addressed in 89\% of the occurrences.
5 or more carbons can be addressed in 57\% of the occurrences
At a magnetic field of 700G more than 3 carbons can be addressed in 90\% of the occurences

% %%%%%%%%%
% Idem:

% Spins that have a stronger than average hyperfine-interaction show up outside or at the edge of the spin-bath collapse.
% Spins that are in the basic regime show up as a narrow dip.
% Going to larger $\tau$ separates these dips further as the order of the resonance $k$ increases.
% By looking at larger $\tau$ it is possible to resolve and address more resonances.
% Several spins in the basic regime have been identified 3 of these are visible as colored lines in \cref{fig:FP}.
% As computations are fundamentally limited by the coherence time there is a limit to the resonance-order that can be used to address carbons, making it impossible to resolve all weakly coupled spins.

% Besides the carbons in the basic regime there are also weakly-coupled carbons that are more strongly coupled.
% When a carbon in the complex regime is present in the NV-center this manifests itself as a resonance with strong oscillations on the side. Such a feature is also clearly visible in \cref{fig:FP}. We have identified the oscillations in the fingerprint as belong to a single spin which is denoted by the red line.

% When a weakly coupled carbon in the complex regime is present a significant part of the fingerprint spectrum is inaccessible for controlling other carbons making them an undesired feature when attempting to control weakly coupled carbon spins.



% Moved from identification section:
% \subsection{Effect of the magnetic field}

% There are significant advantages to increasing the magnetic field when attempting to address weakly coupled carbons.
% By increasing the magnetic field the Larmor frequency can be increased, reducing the number of carbons that are in the complex regime.
% This causes the broad oscillating resonances to disappear allowing more carbons to be addressed.

% Although increasing the magnetic field can improve the situation it is not always possible or desired.
% When the magnetic field becomes too strong too strong the resonances become narrower than the resolution of the Arbitrary Waveform Generator used to generate the pulses that address the resonances, making it impossible to address these resonances effectively.
% Simulations were performed (see \cref{chap:addressable_carbon_sims}) that indicate that for a natural carbon-13 concentration there is a range between 400G and 1400G where the magnetic field is optimal for controlling weakly coupled spins.

% Besides the spin environment there are other factors affecting the choice for magnetic field.
% Because the optical transitions used for readout and initialization depend on strain and magnetic field field\citep{Hensen2011MeasurementBased}, care must be taken when measuring that states do not mix in the excited state.
% This combined with the fact that few experiments have been performed at high magnetic field and low temperature make it more practical to settle for a more moderate magnetic field of 300G.
