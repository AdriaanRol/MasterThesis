\chapter{Towards Quantum Error Correction}
In this thesis we demonstrated the implementation of a parity measurement between two weakly coupled carbons spin.
Two more challenges remain to realize quantum error correction.
First, to correct errors, gate operations that are conditional on the parity outcome are required.
Therefore the parity measurement must be accurate and preserve entangled state for both outcomes \citep{Riste2013Deterministic}, unlike the probabilistic initialization and entanglement in this thesis.

Second, for error correction encoding in larger entangled states is required: at least three qubits for the simplest error correction and at least five qubits for correction of all types of errors.
In this chapter I address these two challenges and present simulations that indicate that multi-qubit parity measurements on weakly coupled spins are possible.

\section{Deterministic Parity Measurements}

An essential capability for QEC is the ability to perform different quantum operations based on a classical readout result.
An example of such an experiment is the deterministic creation of entanglement, as was demonstrated by \citet{Riste2013Deterministic} in a different system.

By applying an extra set of gates when the negative result is measured in a parity measurement the state can be transformed in the state that would have been created had the positive parity been measured.
This capability is particular important for QEC where the result of a parity measurement determines whether a correcting gate must be applied.

The main challenge when performing such operations is to keep track of the phase of the system during the readout that determines which set of gates should be executed.
The measurement environment developed for the work in this thesis automatically keep track of phases and has been set up to be extendable and compatible with such feed forward capabilities.

%Several challenges -> extra phase, Keeping track of things
% An additional advantage of performing deterministic operations is that it can significantly speed up the rate at which experiments can be performed as it is no longer limited by part of the probabilistic initialization.


\section{Multi Qubit Control}
In order to implement QEC more than the two qubits that were addressed are required.
To correct for a single type of quantum error 3 qubits are required.
To correct for any type of quantum error at least 5 qubits are required.


\subsection{Overview of the simulations}
Simulations were performed to determine how many weakly coupled carbons can be controlled, on average, for an NV-center using dynamical decoupling.
In the simulations statistics were collected on 1000 NV-centers for a range of magnetic fields and three different carbon-13 concentrations.
%HOW MANY FOR EACH
For all NV-centers an environment of carbon-13 spins randomly placed on lattice points was generated.
For lower concentrations a larger environment was simulated such that the total number of simulated carbon spins was $\sim 300$ per NV-center for all concentrations.

NV-centers containing a carbon with a hyperfine coupling larger than $200\,\mathrm{kHz}$ were rejected as such spins are prone to decohere during the optical readout of the electron spin \citep{Pfaff2012Demonstration} and their complex dynamics at moderate field complicate spin control of other spins (\cref{chap:addressing_weakly_coupled_carbons}).
In an experimental setting these carbons can be easily detected and the NV-centers rejected.

On the remaining centers the hyperfine coupling between each carbon and the NV-center as well as the resonance conditions of these carbons were calculated.
% 299 for \mu = 1.1%, 423 for 0.33% and even more for 0.011% (data loading crashes..).
At these resonances it was determined if a carbon can be selectively and coherently controlled.

%Definition addressable
As a figure of merit we take the fidelity $F$ of the electron spin after two consecutive $\pm \mathrm{x}$ gates.
The first gate entangles the electron spin with the nuclear spin and the second gate disentangles the two.
Reduced electron spin fidelity indicates gate imperfections or unwanted entanglement of the electron to the other carbon-13 spins.
The threshold for a carbon-13 to be considered controllable is set at $F = 90 \, \%$.

The fidelity is determined by sweeping the number of pulses at the resonance conditions and determining with what fidelity two consecutive $\pm \mathrm{x}$-gates can be implemented.
The accuracy with which the resonance can be addressed is limited by the resolution of the arbitrary waveform generator at $1\,\mathrm{ns}$.
The fidelity is determined under the assumption of perfect $\pi$ and $\pi/2$ pulses on the electron.

%Decoherence implemented
Additionally we require to be able to perform 10 such gates within the coherence time.
For simplicity we do not simulate the nuclear-nuclear interactions and instead set $T_2^*$ as in \cref{tbl:concentration_Tmaxes}.
% interpulse delay.
Finally, we set a maximum value for the inter pulse delay $\tau_{\mathrm{max}}$, to take into account the limited coherence for the electron at large $\tau$.
$\tau_{\mathrm{max}}$ is set at the values listed in \cref{tbl:concentration_Tmaxes}.
%Table listing times used
\begin{table}[htbp]
    \centering
    \caption{Coherence times and maximum resonance time as used in the simulations.}
    \begin{tabular}{ccc}
    Concentration &  $  T_{2,C}^* $ & $  \tau_{\mathrm{max} }$ \\ \hline
    $1.10\,\% $&  7ms & $10\,\mu\mathrm{s}$\\
    $0.33\,\% $&  45ms & $36.7\,\mu\mathrm{s}$\\
    $0.11\,\% $&  70ms & $100\,\mu\mathrm{s}$\\
    \end{tabular}
    \label{tbl:concentration_Tmaxes}
\end{table}

\subsection{Results of the simulation}
\Cref{fig:Simulations_avg_n_vs_bfield} shows the average number of addressable carbon spins as a function of magnetic field.
The number of addressable carbons increases quickly at first before a slow decay sets in.

The initial increase in the number of addressable carbons can be explained by the Larmor frequency becoming larger than the hyperfine interaction.
At low magnetic field ($\omega_L \sim A$) most spins undergo complex dynamics.
When the Larmor frequency becomes larger than the hyperfine coupling of a carbon spin it is in the simple regime where the resonances are narrow.
Because there is less overlap more spins can be addressed.
This increase drops off because eventually all spins are in the simple regime.

By increasing the magnetic field further the resonances both get narrower (\cref{eq:res_dip_width}) and move to shorter times (\cref{eq:res_dip_loc}).
When the field becomes too high a resonance can become too narrow to accurately address.
This is limited by the $1\, \mathrm{ns}$ timing resolution of the arbitrary waveform generator.
Therefore the number of addressable spins decreases at high fields.

For lower concentrations there are on average less spins undergoing complex dynamics for the same magnetic field.
This causes the initial increase in addressable carbons to be steeper at lower concentrations.

There is also a bias due to the rejection of carbons with $|A|>200\,\mathrm{kHz}$.
Not rejecting these carbons will lower the number of addressable carbons for low magnetic fields as the response of these carbons dominates the dynamical decoupling spectroscopy, while the number of addressable carbons will increase slightly in the high field limit were these carbons can be controlled trough dynamical decoupling.

\begin{figure}[htbp]
    \begin{subfigure}[t]{0.49\textwidth}\centering
        \caption{}
        \begin{tikzpicture}
            \node[anchor=south west,inner sep=0] at (0,0) {\includegraphics{Img/Simulations_avgN_vs_Bfield.pdf}};
            \node[font=\small, text = blue] at (5.0,4.90)  {$\mu = 1.10\,\%$};
            \node[font=\small, text = green] at (5.0,4.6)  {$\mu =0.33\,\%$};
            \node[font=\small, text = red] at (5.0,4.30)  {$\mu =0.11\,\%$};
        \end{tikzpicture}
        \label{fig:Simulations_avg_n_vs_bfield}
    \end{subfigure}
    \begin{subfigure}[t]{0.49\textwidth}\centering
    \caption{}
    \begin{tikzpicture}
        \node[anchor=south west,inner sep=0] at (0,0) {\includegraphics{Img/Simulations_Histogram_vs_Bfield.pdf}};
        \node[font=\small, text = black] at (5.4,4.90)  {$B = 700\,\mathrm{G}$};
        \node[font=\small, text = black] at (5.4,4.60)  {$\mu = 1.1\,\%$};
    \end{tikzpicture}
    \label{fig:simulations_histogram_vs_Bfield}
    \end{subfigure}
    \caption{Simulation results calculating the number of addressable carbons in NV-centers containing no carbons with hyperfine coupling larger than $200\, \mathrm{kHz}$. \subref{fig:Simulations_avg_n_vs_bfield} Average number of addressable carbons at different concentrations. \subref{fig:simulations_histogram_vs_Bfield} Normalized occurrences for $\mu = 1.1\,\%$ at $700\,\mathrm{G}$. The simulation indicates that an NV-center likely has 3 or more addressable carbons($P=89\,\%$), that there is a good chance that 5 or more carbons can be addressed ($P=57\,\%$) and that there is even a small probability of getting $\sim 10$ addressable carbon spins. }
    \label{fig:Simulation_results}
\end{figure}



% LONGER COHERENCE TIME
% When the concentration is lower the average separation between carbons and between carbons and the NV- centre is larger. This means that the coherence times are longer and that the coupling between a carbon and the NV- centre is on average weaker.

% OVERLAP
% One would expect that having less carbon atoms means less overlap in the resonances. Having less carbons might mean losing an easy to address carbon that is located close by. This effect is countered by also having some relatively weakly coupled carbons removed reducing overlap in resonances. The exact balance between these two effects depends on the configuration of each individual NV- centre.

% LONGER GATE TIMES
% Because the carbon atoms are coupled more weakly the magnetic field that is required before they can be resonantly addressed is much smaller. The downside of the weaker coupling is that the time it takes to implement a gate is much larger. Longer coherence times allow for longer gate times.

% LONGER RES TIME ALLOWED
% Because the longer coherence time a longer inter-pulse delay is allowed. This allows higher order peaks to be addressed at low magnetic fields. This allows additional carbon atoms to be controlled at low magnetic fields.

% DECLINE
% Eventually the dips become too narrow to address with the AWG. Because the width of the dips (\autoref{eq:res_dip_width}) depends on the strength of the coupling we expect this decline to set in earlier for lower concentrations as the carbons that are addressed are on average weaker coupled.


\paragraph{ }
\Cref{fig:simulations_histogram_vs_Bfield} shows the distribution of addressable carbons for a natural concentration of carbon-13 at the magnetic field that has the highest average number of addressable carbons.
It indicates that in most NV-centers ($89\,\%$) three or more carbons can be addressed and that in a reasonable fraction ($57\,\%$) of the samples five or more carbons can be addressed.
There are even rare occurrences where ten or more carbons can be addressed.

% % %%%%%%%%%
% Idem:

% Spins that have a stronger than average hyperfine-interaction show up outside or at the edge of the spin-bath collapse.
% Spins that are in the basic regime show up as a narrow dip.
% Going to larger $\tau$ separates these dips further as the order of the resonance $k$ increases.
% By looking at larger $\tau$ it is possible to resolve and address more resonances.
% Several spins in the basic regime have been identified 3 of these are visible as colored lines in \cref{fig:FP}.
% As computations are fundamentally limited by the coherence time there is a limit to the resonance-order that can be used to address carbons, making it impossible to resolve all weakly coupled spins.

% Besides the carbons in the basic regime there are also weakly-coupled carbons that are more strongly coupled.
% When a carbon in the complex regime is present in the NV-center this manifests itself as a resonance with strong oscillations on the side. Such a feature is also clearly visible in \cref{fig:FP}. We have identified the oscillations in the fingerprint as belong to a single spin which is denoted by the red line.

% When a weakly coupled carbon in the complex regime is present a significant part of the fingerprint spectrum is inaccessible for controlling other carbons making them an undesired feature when attempting to control weakly coupled carbon spins.



% Moved from identification section:
% \subsection{Effect of the magnetic field}

% Although increasing the magnetic field can improve the situation it is not always possible or desired.
% When the magnetic field becomes too strong too strong the resonances become narrower than the resolution of the Arbitrary Waveform Generator used to generate the pulses that address the resonances, making it impossible to address these resonances effectively.
% Simulations were performed (see \cref{chap:addressable_carbon_sims}) that indicate that for a natural carbon-13 concentration there is a range between 400G and 1400G where the magnetic field is optimal for controlling weakly coupled spins.

% Besides the spin environment there are other factors affecting the choice for magnetic field.
% Because the optical transitions used for readout and initialization depend on strain and magnetic field field\citep{Hensen2011MeasurementBased}, care must be taken when measuring that states do not mix in the excited state.
% This combined with the fact that few experiments have been performed at high magnetic field and low temperature make it more practical to settle for a more moderate magnetic field of 300G.
